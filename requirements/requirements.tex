\chapter{Requirements Capture}

% Projects with a deliverable that serves a specific 
% function often have an initial phase in which expected 
% use is investigated and a brief more detailed than the 
% specification is constructed. This would include what 
% is necessary, what is desirable, etc in the final 
% deliverable. The results of requirements capture 
% determine project objectives and are used to inform 
% project evaluation.
% Requirements capture is important in all projects with 
% real-world deliverables, and is often a significant 
% amount of work in software projects. Where 
% requirements capture is less relevant (for example in
% an analytical ‘research-style’ project) this may be 
% replaced by a detailed description of the project aims 
% and objectives in the Introduction or the Background 
% sections.

\begin{itemize}
    \item Main:
          \begin{itemize}
              \item  Create and evaluate different NN models or SNN models on neuromorphic data from external datasets
              \item Create practical set-up to obtain data from own neuromorphic camera
              \item The main aim of the project is to utilise a neuromorphic camera to carry out SLAM (moving in an unknown trajectory), mapping out the environment around the robot and finding the location it is in.
          \end{itemize}
    \item Fallback:
          \begin{itemize}
              \item  Only using existing datasets rather than practical set-up
              \item Rather than focusing on SLAM the emphasis may be on object detection/recognition
          \end{itemize}
    \item Extensions:
          \begin{itemize}
              \item Carry out some object recognition to localise objects or points of interest in the local environment.
              \item Creating further visualisations of the area map and locations of the robot and objects of interest.
          \end{itemize}
\end{itemize}
