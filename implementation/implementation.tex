\chapter{Initial Implementation} \label{chap:initial_implementation}

\section{Data Preprocessing}

\subsection{Time-relative Event Segmentation}
In order to begin analysing neuromorphic data, it was pre-processed it into a form that a NN can take as input. One such method of doing so was to segment the events into groups based on their timestamp. \autoref{fig:nmnist_spikes_to_intensity_map} shows a visualisation of intensity maps created from the NMNIST\cite{NMNIST} dataset. The set of all events was split into eight segments, where each segment included events within a range of $ 1 \times 10^6 $ ms (i.e., $ 0 \rightarrow 1 $, $ 1 \rightarrow 2 $, ..., $ 7 \rightarrow 8 $). This way the data representation shifted to somewhat get back to a set of frames that mimicked the video output usually seen from everyday cameras. \textbf{(a)} shows the segmented events visualised in three dimensions (x\_location, y\_location and timestamp). In \textbf{(b)} these events were projected onto the two dimensional plane (of x\_location and y\_location), then the plots for on events and off events are shown separately. Finally in \textbf{(c)} an intensity map was created from the projected events. Each pixel in the intensity map grid was initialised to 0, and for every on event 1 was added to the cell, and for every off event 1 was subtracted from the cell. It was clear that the resulting output greatly resembled the MNIST\cite{MNIST} sample recorded by the ATIS camera (As shown in \autoref{fig:nmnist_spikes_visualisation} in \Cref{sec:existing_datasets}).

\begin{figure}[htb]%
    \centering
    \subfloat[\centering]{{\includegraphics[width=0.25\textwidth, height=0.7\textwidth]{implementation/images/nmnist_spikes_visualisation_segmented.png}}}%
    \qquad
    \subfloat[\centering]{{\includegraphics[width=0.4\textwidth, height=0.7\textwidth]{implementation/images/nmnist_events_segmented.png}}}%
    \qquad
    \subfloat[\centering]{{\includegraphics[width=0.25\textwidth, height=0.7\textwidth]{implementation/images/nmnist_events_heatmap_segmented.png}}}%
    \caption{A visualisation of intensity maps created by segmenting events into bins of size $ 1 \times 10^6 $ ms.}%
    \label{fig:nmnist_spikes_to_intensity_map}%
\end{figure}

Once a set of projections was made for each sample from the NMNIST dataset, it could be fed into a neural network in parallel. Ths could be done by simply flattening each intensity map and feeding all the cell values to a fully-connected dense layer in parallel, or the maps could be fed in as images to a convolutional layer. For the convolutional layer method a 3D tensor could be generated by stacking each of the intensity maps against each other and feeding them all directly into the first layer of the NN.

\subsection{Neural Network Input Structure}

There were many ways with which a neural network could accept an input to the system. The ones considered for this project where;

\begin{itemize}
    \item Flattening all the intensity maps from segmented event stream into a vector to be input to a single layer of neurons.
    \item Passing a 3D tensor to an convolutional layer as input.
\end{itemize}

\section{Model Architectures}

Convolutional neural networks have been shown to be much more effective when processing images that networks built solely with dense, fully-connected layers \color{red} add references here \color{black}. This is because they are able to better identify spacial patterns within an image as a kernel spans more than one pixel. For this reason the basic architecture was to have an input layer (the structure of which is dependent on the input format), followed by a series of hidden convolutional layers of varying parameters. Finally the outputs of the hidden layer were fed into a dense layer with 10 neurons to correctly classify the correct class (0-9). Activation functions need to be present in the network to prevent all the layers from becoming equivalent to a single one \color{red} add references here \color{black} (linear regression model). In order to learn more complex patterns activation functions are a necessary aspect of creating an artificial neuron (See \autoref{eq:artificial_neuron_output} in \Cref{ssec:snn_and_heterogeneity}). The most commonly used activation function in deep networks (and image recognition in particular), and in this network as well, is ReLU. \color{red} add references here \color{black}. Finally, the output layer has a sigmoid activation function. This function compresses the output smoothly between the ranges of 0 and 1. this means each of the outputs from the neurons can be interpreted as a the probability of the input being any one of the 10 classes. This means we can simply take the highest probability as the predicted class from the network.

\subsection{Hyper-parameter Tuning}

In order to choose the most appropriate parameters for the system, multiple tests were run varying each of the possible hyper-parameters. The tests conducted were to determine;

\begin{itemize}
    \item The number of hidden layers.
    \item The number of neurons per layer.
    \item The size of convolution kernels.
    \item The introduction of some fully connected layers after convolutional layers.
\end{itemize}

The network in each case was trained for multiple epochs. The performance of a typical network can be seen in \autoref{fig:accuracy_and_loss_per_epoch}, where the performance on the training set steadily improves as the network progresses through epochs. Accuracy is one of the metrics defined in \Cref{chap:evaluation_plan}, and the loss for the given network is called categorical cross-entropy loss. The formula used to calculate the loss is given by: $ L = -\frac{1}{N}\sum^N_{i=1}\sum^C_{c=1}y_c^{(i)}log(\hat{y}_c^{(i)}) $, where there are $ N $ samples and $ C $ classes. $ y_c^{(i)} $ is $ 1 $ when the class is correctly predicted and $ 0 $ otherwise and $ \hat{y}_c^{(i)} $ is the predicted probability of class $ c $ for data-point $ i $.

\begin{figure}[htb]
    \centering
    \includegraphics[width=0.6\textwidth]{implementation/images/accuracy_and_loss_per_epoch.png}
    \caption{A figure showing classification accuracy and cross-entropy loss per epoch on training data for a typical network.}
    \label{fig:accuracy_and_loss_per_epoch}
\end{figure}

It was clear, however, that these results may be misleading since they only represent the efficiency of the system when classifying values within the training data-set. However, when looking at the performance on an unseen test data-set, it is obvious that some of the features learnt do not easily translate to general trends in unseen data. This is known as over-fitting, and can be avoided by reducing the capacity of the data-set so that it does not learn information specific to the training set.

\subsubsection{Convolutional Kernal Size}

\begin{figure}[htb]
    \centering
    \includegraphics[width=0.6\textwidth]{implementation/images/kernel_tests.png}
    \caption{A figure showing classification accuracy and cross-entropy loss on testing data for networks built with varying convolution kernel sizes}
    \label{fig:kernel_tests}
\end{figure}