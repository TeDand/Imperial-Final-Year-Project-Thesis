\documentclass[a4paper, twoside, 11pt]{report}
\linespread{1.1}
%% Language and font encodings
\usepackage[english]{babel}
\usepackage[utf8x]{inputenc}
\usepackage[T1]{fontenc}

%% Sets page size and margins
\usepackage[a4paper,top=3cm,bottom=2cm,left=3cm,right=3cm,marginparwidth=1.75cm]{geometry}

%% Useful packages
\usepackage{amsmath}
\usepackage{subfig}
\usepackage{graphicx}
\setlength {\marginparwidth }{2cm}
\usepackage[colorinlistoftodos]{todonotes}
% \usepackage[colorlinks=true, allcolors=blue]{hyperref}
\usepackage{hyperref}
\usepackage[capitalize]{cleveref} 
\usepackage[numbers]{natbib}
\usepackage{pgfgantt}
\usepackage{ltablex}
\usepackage{multirow}
\usepackage{color, colortbl}

% \usepackage{tikz}
% \usepackage{pgfplots}

\usepackage{listings}
\crefname{lstlisting}{listing}{listings}
\Crefname{lstlisting}{Listing}{Listings}
\usepackage{xcolor}

\definecolor{codegreen}{rgb}{0,0.6,0}
\definecolor{codegray}{rgb}{0.5,0.5,0.5}
\definecolor{codepurple}{rgb}{0.58,0,0.82}

\lstdefinestyle{mystyle}{ 
    commentstyle=\color{codegreen},
    keywordstyle=\color{magenta},
    numberstyle=\tiny\color{codegray},
    stringstyle=\color{codepurple},
    basicstyle=\ttfamily\fontsize{8}{8}\selectfont,
    breakatwhitespace=false,         
    breaklines=true,                 
    captionpos=b,                    
    keepspaces=true,                 
    numbers=left,                    
    numbersep=5pt,                  
    showspaces=false,                
    showstringspaces=false,
    showtabs=false,                  
    tabsize=2,
    xleftmargin=.12\textwidth
}

\lstset{style=mystyle}

\title{Data-driven Classification Methods for Event-driven Cameras}
\author{Tejas Dandawate}
% Update supervisor and other title stuff in title/title.tex

\begin{document}
\input{title/title.tex}

\renewcommand{\abstractname}{Final Report Plagiarism Statement}
\begin{abstract}
    I affirm that I have submitted, or will submit, an electronic copy of my final year project report to the provided EEE link.

    I affirm that I have submitted, or will submit, an identical electronic copy of my final year project to the provided Blackboard module for Plagiarism checking.

    I affirm that I have provided explicit references for all the material in my Final Report that is not authored by me, but is represented as my own work.
\end{abstract}

\renewcommand{\abstractname}{Acknowledgements}
\begin{abstract}
    Firstly, I would like to thank Prof. Pier Luigi Dragotti for the opportunity to undertake this project under his supervision. He gave me a lot of much needed guidance throughout the process, while still affording me the freedom to take the project where I saw fit.

    I would also like to extend my gratitude to my family without whom I would not be in this position, I will be forever in their debt. And finally to my friends, who have been instrumental in keeping me sane through these four long years - thank you all.
\end{abstract}

\renewcommand{\abstractname}{Abstract}
\begin{abstract}
    Neuromorphic sensing is a novel way of encoding analogue signals, inspired by the biological processing of information in our brains. Neuromorphic sensing is based on time encoding, where rather than recording the amplitude of the input signal at predefined times, one records the time instants where the amplitude surpasses a certain trigger mark. This leads to a very efficient way of acquiring and processing signals.

    The main aim of the project is to utilise data retrieved from neuromorphic cameras to carry out video classification. Two pipelines are devised to achieve this; one involves directly classifying event data, while the other is a two-phase network were events are first reconstructed into intensity videos before classification is done. With both pipelines, a series of networks are used for classification in order to ensure there is a fair comparison.
    
    Data pre-processing is also a large part of the machine learning pipeline, and as such played a part in the creation of the two pipelines. For the direct event classification pipeline in particular, varying pre-processing methods are used. In most neuromorphic data analyses a method called `frame integration' is utilised, where events are segmented into time bins and accumulated into frames. In the classical approach, each frame has two channels depicting whether a positive of negative event occurred in each pixel during a particular time bin. In this project, the classical implementation was used as well as a new novel approach. Multiple networks were designed and implemented to be a part of both classification pipelines. The main ones  were convolutional and recurrent neural networks. Finally, a fully neuromorphic system using spiking neural networks is devised, allowing for the exploitation of the sparse data communication between layers. With similar performance, these networks can be utilised for low-powered devices and real-time classification in future works.
\end{abstract}

\pagenumbering{gobble}

\tableofcontents
\listoffigures
\listoftables \thispagestyle{empty}

\pagenumbering{arabic}

\chapter{Introduction}
% The introduction should set the scene and give a high-level problem statement/specification, so that after 
% reading the introduction the reader understands
% roughly what the problem is and what you intend to 
% do about it. Is the idea to write software, or develop 
% an algorithm, or produce hardware, or something 
% else?
% You should then highlight and summarise the most 
% interesting or important questions or problems that 
% your project addresses, and the broader context in 
% which those questions or problems are situated.
% Finally, you must briefly introduce the structure of 
% report (what you will cover in which chapters and how 
% these relate to each other). You don’t need to go into 
% any detail, the aim is to make sure the reader has an 
% idea about what will be discussed and in what order.

\section{Motivations}

Neuromorphic data is a novel representation of data that has overarching benefits that are yet to be fully explored in a plethora of applications. The many purported benefits of event-based cameras have the potential to revolutionise efficient, low-power computer vision due to their efficient encoding of information. Since the cameras are asynchronous in nature they allow for low-latency feeds with very little motion blur and other similar visual artefacts. As well as this their pixel structure allows for a very high dynamic range making the uses of these cameras even more apparent.

With the emergence of the IoT, sensors have become, or at the very least will become, ubiquitous in everyday life. These devices operate using power at a premium since they need to make full use of their limited batteries. This scarcity is further emphasised when attempting to do more power-intensive tasks. One such task is computer vision, which is becoming evermore prevalent as a means of human-computer interaction. Classical frame-based cameras are very power intensive, and do not allow for a low active duty cycle (i.e., allowing for device sleep/idle time). The usual way this issue is alleviated it by the system reacting to an event trigger. Such events may include things such as; motion, timing, acceleration or temperature. When compared to the function of event-based cameras, where event detection in built into the system, this can be seen as a stop-gap measure.

The high temporal resolution and dynamic range of event-based cameras also presents further benefits when compared to frame-based cameras. These features allow for videos to be represented in a very high fidelity format that preserves more data in fast-paced and brightly lit environments. This may lead to more reliable use of the data for computer vision and even other purposes.

\section{Objectives}

The objectives of this project were roughly divided into three two main sections; main objectives and extensions.

\subsection{Main Objectives}

\begin{itemize}
    \item Evaluate different Artificial Neural Network (ANN) models on neuromorphic data from external datasets.

          This allows for determining the performance of both traditional Neural Networks (NNs) as well the performance Spiking Neural Networks (SNNs) for comparison. These networks can take two main forms:
          \begin{itemize}
              \item Two networks sequentially processing data. The first would be converting reconstructing spiking data into an intensity video, and the second would be applying pre-existing computer vision networks to analyse the frame-based output.
              \item Having one network that takes the spiking data as input to directly carry out the intended function.
          \end{itemize}
          The system will initially be used to solve a classification task.
    \item Create practical set-up to experimentally record data.

          Using a neuromorphic camera, data can be experimentally captured to assess the performance of any built networks on more realistic unseen input data.
    \item Carry out Simultaneous Localisation and Mapping (SLAM) for a robot moving in an unknown trajectory.

          Use most efficient NN model to solve the more complex task of SLAM. The neuromorphic camera can be used to capture data from a room in the absence of large sets of labelled datasets.
\end{itemize}

\subsection{Challenges \& Fall-backs}

\begin{itemize}
    \item Only using existing datasets rather than practical set-up.

          Since there is sufficient amounts of existing data-sets, they can be split into training, validation and testing sets themselves. This eliminates the need to generate more data for the testing process.
    \item Rather than focusing on SLAM the emphasis may be on object detection/recognition

          The project has be segmented into individual milestones, and so even if the final one of a SLAM algorithm isn't achieved, it is easy to change the scope of the project to be simply a classification or gesture recognition network.
\end{itemize}

\section{Report Structure}

A brief outline of the report is as outlined:

\begin{enumerate}
    \item Background

          This section gives a background to the project subject. There are outlines of previous works in the field in order to highlight gaps in knowledge where more work can be done. It provides a good illustration of what issues there are and what value there would be in solving them.
    \item Implementation Plan

          This section has a list of planned objectives and their estimated completion date.
    \item Evaluation Plan

          This section presents a variety of ways in which the final project could be evaluated at each step. This allows for a comparison between existing solutions as well every iteration of solution given during the project.
    \item Ethical, Legal and Safety Plan

          Any ethical, legal and safety considerations taken into account during the project are given in this section.
\end{enumerate}
\chapter{Background}

% You should provide enough background to the reader 
% for them to understand what the project is all about, 
% and what is the relevant prior work.
% Examiners like to know that you have done the 
% appropriate background research and it is important 
% that you review either what has been done previously 
% to tackle related problems, or perhaps what other 
% products exist related to your deliverable. Clear 
% references are important here, and much of this 
% section will typically already have been written in your 
% Interim Report. You may use feedback from that 
% report to improve what you write in your Final Report, 
% and should note that self-plagiarism between the two 
% reports is not possible, so no citation is needed of 
% your own earlier writing.

%  What does the reader need to know in order 
% to understand the rest of the report? What 
% problem are you solving?
%  Why is this problem interesting or worthwhile 
% to solve?
%  Who cares if you solve it?
%  How does this relate to other work in this 
% area?
%  What work does it build on?
%  For 'research-style' projects involving the 
% design and analysis of specific algorithms 
% there is a large amount of relevant 
% background both of general theory, and very 
% specific to the algorithm you investigate. 
% Supervisors will help you to see what is most 
% important here, but the general rule is that 
% you must both provide overall context and 
% note work close to what you do that 
% influences your work or is in some way 
% comparable to your work.

This chapter outlines background information required for understanding the basis for the project. The theory and literature serves to outline the main concepts used for neuromorphic SLAM, as well as to reveal gaps in existing research that require solidifying.

\section{Event Cameras}

Event based cameras can be described as `bio-inspired sensors that differ from conventional frame cameras: Instead of capturing images at a fixed rate, they asynchronously measure per-pixel brightness changes, and output a stream of events that encode the time, location and sign of the brightness changes' \cite{EventBasedVisionASurvery}.

\subsection{Benefits}
Event-based cameras are purported to provide a number of benefits including, but not limited to;

\begin{itemize}
      \item \textbf{Very high temporal resolution}

            The reason for this is that whereas frame-based cameras have a certain frame-rate, event-based cameras do not have this limitation, meaning the "blind time" between frames is eliminated. The reason for this is that the function of a frame-based camera is dependent on the global shutter to capture the light at a particular instant, whereas event-based cameras can be thought of as having individual shutters for each pixel that are shut whenever an even occurs.
      \item \textbf{High dynamic range}

            The reason for this is again the fact that each pixel has its own individual shutter, but as well as this they all use a logarithmic scale, meaning they function well from very bright to very dim environments as well as fast shifts between the two.
      \item \textbf{Low power consumption}
      \item \textbf{High pixel bandwidth}

            Each pixel can capture events at the rate of ~kHz. This has the effect of reducing blur since there is a very high temporal resolution to begin with. This makes the system very responsive and therefore ideal for real-time systems.
      \item \textbf{Efficient Encoding}

            Since events are asynchronous and spatially sparse (i.e there are mainly 0 values in the matrix), the encoding is very efficient, as opposed to frame-based cameras that produce data that is very spatially dense.
\end{itemize}

The above benefits are very persuasive reasons to adopt neuromorphic cameras in many different applications. It is conceivable that if algorithms can make use of these benefits (since most classical algorithms play to the strengths of the data generated by frame-based cameras), real-time systems could be completely revolutionised.

\subsection{Function}
Event-based cameras differ from frame based cameras fundamentally, in that they do not rely on a global shutter closing at regular intervals to record information of a scene. Instead each pixel closes whenever it detects an 'event' occurring. The way to detects such events are dictated by the `event generation model'\cite{EventBasedVisionASurvery}.

Each pixel responds to changes in its log photo-current ($ L = log(I) $, where $I$ is the perceived brightness), giving the system a very high dynamic range. A recorded event '$ k $' has the format $ e_k = (\boldsymbol{\mathbf{x}}_k, t_l, p_k ) $. THis is known as the Address Event Representation (AER). The first value is the spacial location of the event ($ \boldsymbol{\mathbf{x}}_k = (x_k, y_k)^\top $), the second value $ t_k $ is the temporal location, and the final value $ p_k \in {1, -1} $ indicates the polarity of the event (i.e in which direction the brightness gradient was changing). The brightness increment between two events at the same pixel is given by the equation $ \Delta L(\boldsymbol{\mathbf{x}}_k, t_k) = L(\boldsymbol{\mathbf{x}}_k, t_k) - L(\boldsymbol{\mathbf{x}}_k, t_k - \Delta t_k) $. In a perfect (noise free) environment an event is fired whenever the brightness increment reaches a temporal contrast threshold given by the equation $ \Delta L(\boldsymbol{\mathbf{x}}_k, t_k) = p_k C $ ($ C > 0 $). It should be noted that the value of $ C $ could be variable and therefore different for $ p_k = \pm 1 $.

Additionally, we can approximate the temporal derivative of a pixels brightness by utilising the following Taylor expansion:

$$ \Delta L(\boldsymbol{\mathbf{x}}_k, t_k) \approx \frac{\delta L}{\delta t}(\boldsymbol{\mathbf{x}}_k, t_k)\Delta t_k $$
$$ \frac{\delta L}{\delta t}(\boldsymbol{\mathbf{x}}_k, t_k) \approx \frac{\Delta L(\boldsymbol{\mathbf{x}}_k, t_k)}{\Delta t_k} = \frac{p_k C}{\Delta t_k} $$

The above approximation, however, is only true under the assumption that $ \Delta t_k $ is exceedingly small. Since unlike frame-based cameras we do not measure absolute brightness, this is an indirect way of measuring and keeping track of the brightness within the frame.

\autoref{fig:davis_camera} shows the basic functionality of an event based camera. \textbf{(a)} is the simplified circuit diagram of the DAVIS pixel, which in \textbf{(b)} is used to convert light into events (shown in real life in images \textbf{(c)} and \textbf{(d)}). \textbf{(e)} shows how this setup would view a white square rotating on a black disk. It is a stream of events going from the past in green to the present in red. These events can then be seen overlaid on a natural scene in \textbf{(f)}.

\begin{figure}[htb]
      \centering
      \includegraphics[width=0.7\textwidth]{background/images/davis_camera.png}
      \caption{Summary of DAVIS event based camera\cite{EventBasedVisionASurvery}}
      \label{fig:davis_camera}
\end{figure}

\subsection{Optic-flow Methods}

More classic computer vision techniques using optic-flow constraints can now be utilised to characterise the events detected by pixels. In frame-based systems, optic flow methods create a flow-field that describe the displacement vector (signifying direction and magnitude of movement) for each pixel in the frame. A core constraint in this derivation is that the intensity of a local time-varying image region is constant under motion (for at least a short amount of time)\cite{GenerativeEventModel}.

\color{red} NEED TO READ AND UNDERSTAND WHY THE FOLLOWING IS TRUE \cite{GenerativeEventModel} \cite{EventBasedVisionASurvery} \color{black}

$$ \Delta L \approx -\nabla L \cdot v \Delta t_k $$

\section{Neural Heterogeneity}

Above are very persuasive reasons for utilising neuromorphic systems, but there still many challenges posed when attempting to do so. For example, each pixel only responds to brightness change, but the problem is that such a change could be a result of not only scene changes, but also the position of the camera within the scene. For this reason most neuromorphic systems have currently been limited to stationary cameras. As well as this the system is especially prone to stochastic noise, due to inherent shot noise in photons and from transistor circuit noise \cite{EventBasedVisionASurvery}. In order to tackle such issues, it is useful to look at existing examples of spiking neural systems, such as biological brain with neurons working based on a spiking function. It is known that the brain is heterogeneous on every scale, in the past this was though to be simply a by-product of noisy processes, but more recently it can be shown that by evolving our largely homogeneous spiking neural networks (SNNs), we can create more stable and robust systems \cite{NeuralHetroPromRobLearn}.

\color{red} NEED TO READ EVENT PAPER TO SEE THEIR APPROACH TO STOCHASTIC NOISE \cite{EventBasedVisionASurvery}. ALSO READ PAPER\cite{NeuralHetroPromRobLearn} TO INCLUDE IMAGE OF AND EXPLANATION OF BRAIN NEURONS \color{black}

\section{Existing Algorithms for Event Analysis}

For SLAM and pose estimation the problem again is that classical systems heavily rely on the structure of conventional cameras, and so there needs to be a radical paradigm shift in order to take events as inputs instead. The reason for this is that their function depends on iterative changes to the location probabilities using inputs, for which spiking inputs are ideal.

\subsection{Probabilistic Filters}

\subsubsection{Bayesian Inference}

Unlike most other previous systems, probabilistic filters such as Bayesian filters inherently work for the new scenario with event-based cameras. Bayes's theorem can be derived from simple probabilistic rules\cite{BayesLaw}. We know $P(X|Y) = \frac{P(X, Y}{P(Y)} $ and similarly $ P(Y|X) = \frac{P(Y, X)}{P(X)} $. Therefore we can re-arrange both to give $ P(X|Y)P(Y) = P(Y|X)P(X) $, since $ P(X, Y) = P(Y, X) $. Then from there formula for Bayesian inference can be trivially obtained:

$$ P(XZ) = P(Z|X)P(X) = P(X|Z)P(Z) $$
$$ P(X|Z) = \frac{P(Z|X)P(X)}{P(Z)}$$

In the above equations $ X $ is known as the prior (which is the assumed location of the camera) and $ Z $ is known as the posterior (which is the measurement taken by the sensor or camera). $P(Z|X) $ is known as the likelihood function, which indicates how likely it is to have received the particular reading given the assumed position.

\subsubsection{Monte-Carlo Localisation}

Now that we have the concept of Bayesian inference we can adapt it to create an efficient localisation algorithm. It includes initialising a number of particles that act as predictors of where in the map the camera is. We can now give the probability of a posterior camera position given a sensor reading.

The probability distribution $ P(X|X) $ is a continuous function, and so updating each of the posteriors for every value of $ X $ is a computationally difficult problem. We can instead break it up into smaller bins to alleviate this issue. When generating these particles we can represent the probability distribution, albeit at a lower granularity. The benefit of this is that even though we cannot see the full distribution the peaks (i.e. the locations the camera is most likely to be in) are very well defined.

Now we can use the above simplification to carry out the following steps:

\begin{enumerate}
      \item Randomly assign particle distribution across map
      \item Apply Bayes' law to measurement to update particle distribution

            Bayes' rule can be simplified to be:
            $$ w_{i_{x+1}} = P(z|x_{i_x}) \times x_{i_x} $$
            This can be done since the ignored multiplier on the right hand side will be normalised in the next step.
      \item Normalise particle weights

            Now the particles will have weights that no longer sum to 1, and so we need to normalise them again to follow the usual rules of probability:

            $$ w_{i_{x+1}} = \frac{w_i}{\sum^N_{i=1}w_i} $$
      \item Re-sample particle distribution

            We now need to create a new set of particles that all have the same weight ($ \frac{1}{N} $, but whose spacial distribution now reflects the probability density.
\end{enumerate}

\begin{figure}[htb]
      \centering
      \includegraphics[width=\textwidth]{background/images/monte_carlo.png}
      \caption{Example of Monte-Carlo localisation\cite{MonteCarloLocalisation}}
      \label{fig:monte_carlo}
\end{figure}

\autoref{fig:monte_carlo} shows a typical example of the algorithm. The leftmost panel shows the random initialisation (or previous particle distribution), which then becomes the centrally shown distribution after one iteration. Since only one measurement is taken and the room is symmetrical it is possible that it could be in one of two location (hence the two dense clusters). After one more reading in the next iteration the algorithm is quickly able to narrow down the location of the robot.

This algorithm, however, is classical and has therefore been mostly superseded by deep neural networks in most modern day applications. As well as this it is only applicable to localising the robot, whereas we want to be able to simultaneously map it. SLAM can be achieved using similar methods by identifying and moving around located points of interest and estimating their positions as well as the robots. These tasks have been efficiently solved by neural networks for classical frame based data, but there is still much ongoing research on how to do the same with spiking data.

\color{red} NEED TO FIND CLASSICAL EXAMPLE OF SLAM ALGORITHM TO REFERENCE AS WELL AS IN DEPTH EXPLANATION OF LOOP CLOSURE ETC. \color{black}

\section{Computational Neuroscience}

The foundations of Spiking Neural Networks (SNNs) are in computational neuroscience. The mechanisms of neurons in the brain are the inspiration behind creating artificial neural networks with neurons that spike in the same way. Neurons in an a typical Artificial Neural Network (ANN) have a weight, bias and activation function. this means the output of the neuron can be specified as:

$$ y = \theta(\sum^n_{j=1}w_jx_j-u_j) $$

\color{red} WRITE SOME STUFF FROM COURSERA COURSE\cite{RaoCoursera} \color{black}

\section{Existing Datasets}

There already exists many repositories of recorded neuromorphic data to get familiar with spiking data.

\subsection{Neuromorphic MNIST}

This data-set is a spiking version of the original frame-based MNIST dataset \cite{NMNIST}. The dataset is identical to the original MNIST dataset in all ways (including scale, size and sample split) bar one - it was captured using an ATIS sensor mounted on a motorised pan-tilt unit. This sensor moved while viewing the MNIST examples on an external monitor.

For each item in the dataset there is a binary file which has a list of events. Each event is characterised by a 40 bit unsigned integer. The integer gives the following information of a particular event:

\begin{itemize}
      \item bit 39 - 32: X location (in pixels)
      \item bit 31 - 24: Y location (in pixels)
      \item bit 23: Polarity (0 for OFF, 1 for ON)
      \item bit 22 - 0: Timestamp (in microseconds)
\end{itemize}

More datasets include"

\begin{itemize}
      \item DVS128
      \item Other data-sets such as fashion MNIST could also be converted to spiking times by treating image intensities as input currents to model neurons, so that higher intensity pixels would lead to earlier spikes, and lower intensity to later spikes
      \item Heridelberg Spiking data-sets (SHD and SSC)
\end{itemize}

\section{Image Reconstruction Algorithms}

Image reconstruction has been implemented for event data using on the direct optimised versions of Convolutional Neural Networks (CNNs). An example of this is the network named 'U-net'\cite{UNET} which managed to reconstruct a video using 10M parameters to analyse events from an AER camera protocol. Recent work by Rebecq \textit{et al.} illustrates a novel network architecture that reconstructs a video from a stream of events \cite{spikingToVideo}. These methods are purported to allow the introduction of mainstream computer vision research to event cameras. \autoref{fig:spikes_to_video} shows an example of how converting spiking data to a video stream allows for use of classical computer vision algorithms.

\begin{figure}[htb]
      \centering
      \includegraphics[width=\textwidth]{background/images/spikes_to_video.png}
      \caption{Illustration of mapping of spiking data to video stream to apply off-the-shelf algorithms to\cite{spikingToVideo}}
      \label{fig:spikes_to_video}
\end{figure}

A naive approach would be take take each would be to take each event $ e_k = (\boldsymbol{\mathbf{x}}_k, t_l, p_k ) $ and assuming that the firing was due to the brightness change was due to a brightness change above a threshold $ \pm C $ which is a constant that could be set by the user. If this was the case events could be directly integrated to recover the intensity map of images. however, the value $ C $ in reality does not remain constant and is heavily dependent on other factors such as event rate, temperature, and sign of brightness change. The implementation outlined instead makes use of a Recurrent Neural Network (RNN), that takes as input sets of events within a spatio-temporal window. For example, a stream of events will be broken down into sequences given by $ \epsilon_i \: \forall i \in [0, N-1] $. Since each sequence is of fixed length $ N $ the framerate of the output video from the RNN is proportional to the event rate. \autoref{fig:spikes_to_video_rnn} shows demonstrates the functionality of such a network. Each event window $ \epsilon_k $ is converted to a 3D event tensor and passed into the network along with the last $ K $ constructed images to generate the latest iteration of the image. It is clear from this that each new image is constructed by fusing the previous K images with the new stream of events.

\begin{figure}[htb]
      \centering
      \includegraphics[width=\textwidth]{background/images/spikes_to_video_rnn.png}
      \caption{Overview of RNN used to generate video from sets of events\cite{spikingToVideo}}
      \label{fig:spikes_to_video_rnn}
\end{figure}
% \chapter{Implementation Plan} \label{chap:implementation_plan}

\section{Timeline}

% oct 22 - jun 6
\begin{figure}[htb]
    \centering
    \begin{ganttchart}[x unit=0.5cm,
            y unit title=0.5cm,
            y unit chart=0.6cm]{1}{18}
        % \gantttitle{Project Timeline}{18} \\
        \gantttitlelist{1,...,18}{1} \\
        \ganttgroup{Inception Report}{1}{2} \\
        \ganttbar{Summary of Deliverables}{1}{1} \\
        \ganttlinkedbar{Background Research Plan}{2}{2} \ganttnewline
        \ganttgroup{Interim Report}{3}{8} \\
        \ganttbar{Project Specification}{3}{3} \\
        \ganttlinkedbar{Background Research}{4}{6} \\
        \ganttlinkedbar{Implementation Plan}{7}{8} \\
        \ganttbar{Evaluation Plan}{7}{8} \\
        \ganttbar{Ethical, Legal, Safety Plan}{7}{8} \\
        \ganttgroup{Final Report}{9}{16} \\
        \ganttbar{Understanding Data}{9}{9} \\
        \ganttlinkedbar{Create Sequantial NN System}{10}{10} \\
        \ganttlinkedbar{Create Parallel NN System}{11}{11} \\
        \ganttmilestone{Classification}{11} \ganttnewline
        \ganttlink{elem12}{elem13}
        \ganttbar{Set-up Camera}{12}{12} \\
        \ganttlinkedbar{Test System on Camera Data}{13}{13} \\
        \ganttmilestone{Experimental Setup}{13} \ganttnewline
        \ganttlink{elem15}{elem16}
        \ganttlinkedbar{Convert System for SLAM}{14}{16} \ganttnewline
        \ganttbar{Evaluate Created Systems}{14}{16}  \\
        \ganttgroup{Final Presentation}{17}{18} \\
        \ganttbar{Prepare Presentation}{17}{18} \\
        \ganttbar{Proof Reading}{17}{18}
    \end{ganttchart}
    \caption{A gantt chart showing fortnightly progress. Main deliverables are written in \textbf{bold}, and milestones are written in \textit{italics}.}
    \label{fig:gantt_chart}
\end{figure}

\newpage
\section{Objectives}

% oct 22 - jun 6
\definecolor{Gray}{gray}{0.9}
\newcolumntype{P}[1]{>{\raggedright\let\newline\\\arraybackslash\hspace{0pt}}p{#1}}
% \begin{figure}
\begin{longtable}{|| P{0.2\textwidth} | P{.45\textwidth} | P{.15\textwidth} ||}
    \hline
    \rowcolor{Gray} \multicolumn{1}{|| c}{\textbf{Name}} & \multicolumn{1}{| c |}{\textbf{Description}}                                                                                                                                                                                                                                                                                                                                                                                      & \multicolumn{1}{c ||}{\textbf{Timeline} } \\
    \hline \hline \endhead
    Summary of Deliverables                              & Create an initial plan of deliverables to be present in final project. As well as this it is important to have some initial plans for fallbacks to ensure that a complete project can be achieved.                                                                                                                                                                                                                                & 22/10/2021 $ \rightarrow $ 5/11/2021      \\
    \hline
    Background Research Plan                             & Create an initial list of relevant papers to kickstart the project.                                                                                                                                                                                                                                                                                                                                                               & 5/11/2021 $ \rightarrow $ 19/11/2021      \\
    \hline
    Project Specification                                & Create a specification that precisely defines the goals of project as well as fallbacks for each case.                                                                                                                                                                                                                                                                                                                            & 19/11/2021 $ \rightarrow $ 3/12/2021      \\
    \hline
    Background Research                                  & Use initial list of papers to conduct research behind the topic of the project. This involves looking at previous works and their respective gaps in knowledge. This is in order to give a basis on which to begin implementation, as it should provide all necessary tools and knowledge to begin initial preparations. It is also useful so give an  insight on the need for the project and what problems it aims to overcome. & 3/12/2021 $ \rightarrow $ 14/01/2022      \\
    \hline
    Implementation Plan                                  & Using analysis from background reading create a structured implementation plan outlining objectives and milestones, including a deadline for each.                                                                                                                                                                                                                                                                                & 14/01/2022 $ \rightarrow $ 31/01/2022     \\
    \hline
    Evaluation Plan                                      & Create a structure for the evaluation of any implemented material, including detailed explanations for any formulae and significance of any values when compared to baselines.                                                                                                                                                                                                                                                    & 14/01/2022 $ \rightarrow $ 31/01/2022     \\
    \hline
    Ethical, Legal, Safety Plan                          & Critically evalutate any ethical, legal or safety concerns that may arise as a result of this project and outline various ways in which to mitigate any possible issues.                                                                                                                                                                                                                                                          & 14/01/2022 $ \rightarrow $ 31/01/2022     \\
    \hline
    Understanding Data                                   & Load and evaluate the utility of various data sources and pre-process them ready for use in future NNs.                                                                                                                                                                                                                                                                                                                           & 31/01/2022 $ \rightarrow $ 25/02/2022     \\
    \hline
    Create Sequential NN System                          & Create model that first reconstructs frame-based video from neuromorphic data to then pass into adaptations of existing networks that carry out classification on frame-based videos.                                                                                                                                                                                                                                             & 25/02/2022 $ \rightarrow $ 11/03/2022     \\
    \hline
    Create Parallel NN System                            & Create a single NN that takes neuromorphic data and directly creates the required output (classification or SLAM).                                                                                                                                                                                                                                                                                                                & 11/03/2022 $ \rightarrow $ 25/03/2022     \\
    \hline
    Set-up Camera                                        & Create set-up to obtain real-world data from event-based camera. This also required to create a pipeline to process data to be used in created NNs.                                                                                                                                                                                                                                                                               & 25/03/2022 $ \rightarrow $ 08/04/2022     \\
    \hline
    Test System on Camera Data                           & Since real world data may be less idealised than pre-existing datasets, it is important to test the functionality of the system with data obtained from camera. As well as this tests can be done under various more extreme circumstances to test robustness of system.                                                                                                                                                          & 08/04/2022 $ \rightarrow $ 22/04/2022     \\
    \hline
    Convert System for SLAM                              & Adapt function of system and train for SLAM rather than classification.                                                                                                                                                                                                                                                                                                                                                           & 22/04/2022 $ \rightarrow $ 29/05/2022     \\
    \hline
    Evaluate Created Systems                             & Use metrics listed in the evaluation plan to check the performance of each created system in order to compare them to each other as well as other existing solutions from other researchers.                                                                                                                                                                                                                                      & 22/04/2022 $ \rightarrow $ 29/05/2022     \\
    \hline
    Prepare Presentation                                 & Create slides and content to present the final project results.                                                                                                                                                                                                                                                                                                                                                                   & 29/05/2022 $ \rightarrow $ 06/06/2022     \\
    \hline
    Proof Reading                                        & Check through the report and presentation to remove errors and make any last-minute changes.                                                                                                                                                                                                                                                                                                                                      & 29/05/2022 $ \rightarrow $ 06/06/2022     \\
    \hline
    \caption{A table of explanations of objectives given in \autoref{fig:gantt_chart}.}
    \label{tab:objectives_table}
\end{longtable}
% \chapter{Evaluation Plan} \label{chap:evaluation_plan}

The evaluation plan is as given by the typical machine learning pipeline\cite{IntroToML}.

\section{Dataset Preparations}

\subsection{Dataset Splitting}

It is commonplace to split the shuffled dataset into  three segments; training, validation and testing. The training data is what is used during each iteration of the back-propagation process. The validation data is what is unseen during this process and is instead used to give an estimation of a models performance while training hyper-parameters. Finally, the testing data is withheld until it is needed to compare different final implementations with each other. It is vital the the training and validation data are withheld while training since otherwise they would not serve as a simulation of unknown data to measure the performance of the system in an unbiased manner. Common splits for training, validation and testing datasets are 60\%/20\%/20\% and 80\%/10\%/10\%  respectively.

\subsection{Dataset Cross-validation}

With smaller datasets the splitting of data may mean that there is too little left to train with. This problem can be alleviated by using cross-validation. This method entails dividing the dataset into a certain number of segments. Then in the first iteration of the learning process, the first segment is used as testing data while the rest is used as training and validation. Then in the next iteration the process can be repeated by using the second segment as the testing, and so on until each segment has been used as testing data. Finally we can use the average of the errors for each testing dataset as the `global error estimate'. It can be noted that the same segmentation and iteration process can be used for the training and validation datasets.

\section{Evaluation Metrics}

For a classification task, when we obtain the results from the test dataset (as shown in \autoref{tab:possible_results}) we can calculate a variety of evaluation metrics that give various insights on our final model.

\begin{table}[htb]
    \centering
    \begin{tabular}{|| c  | c ||}
        \hline
        Labels     & Predictions \\
        \hline \hline
        1          & 1           \\
        \hline
        1          & 2           \\
        \hline
        3          & 8           \\
        \hline
        9          & 9           \\
        \hline
        6          & 9           \\
        \hline
        $ \vdots $ & $ \vdots $  \\
    \end{tabular}
    \caption{A table showing an example of results when inputting test data from NMNIST dataset\cite{NMNIST} into the final model.}
    \label{tab:possible_results}
\end{table}

\subsection{Confusion matrix}

Confusion matrices act as a visualisation of a systems performance. It shows possible true labels as well as possible predicted labels on either side, and filled in are the number of results that fit in each segment. In \autoref{tab:confusion_matrix} the confusion matrix for the NMNIST dataset is shown as an example. It should be noted that a similar confusion matrix should be created taking each class as positive, then each metric can be calculated by taking the averages (as shown in \Cref{ssec:eval_metric_averaging}). For each of the cells the number of matching records are stored to calculate each of the evaluation metrics. The table includes True Positives (TP), False Positives (FP), True Negatives (TN) and False Negatives (FN).

\begin{table}[htb]
    \centering
    \begin{tabular}{|| c c | c | c | c | c ||}
        \hline
                                                                         &                                    & \multicolumn{4}{ c ||}{\textbf{Predicted Class}}                                        \\
        \cline{3-6}
                                                                         &                                    & 1                                                & 2          & 3          & $ \hdots $ \\
        \hline
        \multirow{6}{*}{\rotatebox[origin=c]{90}{\textbf{Actual Class}}} & \multicolumn{1}{| c |}{1}          & TP                                               & FN         & FN         & $ \hdots $ \\
        \cline{2-6}
                                                                         & \multicolumn{1}{| c |}{2}          & FP                                               & TN         & TN         & $ \hdots $ \\
        \cline{2-6}
                                                                         & \multicolumn{1}{| c |}{3}          & FP                                               & TN         & TN         & $ \hdots $ \\
        \cline{2-6}
                                                                         & \multicolumn{1}{| c |}{4}          & FP                                               & TN         & TN         & $ \hdots $ \\
        \cline{2-6}
                                                                         & \multicolumn{1}{| c |}{5}          & FP                                               & TN         & TN         & $ \hdots $ \\
        \cline{2-6}
                                                                         & \multicolumn{1}{| c |}{$ \vdots $} & $ \vdots $                                       & $ \vdots $ & $ \vdots $ & $ \ddots $ \\
    \end{tabular}
    \caption{a table showing one particular confusion matrix for NMNIST dataset\cite{NMNIST} for class 1 as the positive class.}
    \label{tab:confusion_matrix}
\end{table}

\subsection{Accuracy}

The accuracy of the system is the proportion of samples correctly classified.

$$ Accuracy = \frac{TP + TN}{TP + TN + FP + FN} $$

Note: classification error can also be used and is defined as $ 1 - accuracy $.

\subsection{Precision}

Precision is the proportion of positively predicted samples identified correctly.

$$ Precision = \frac{TP}{TP + FP} $$

It should be noted that a high precision may mean that there are many false positives.

\subsection{Recall}

Recall is the proportion of actual positives correctly classified.

$$ Recall = \frac{TP}{TP + FN} $$

It should be noted that a high recall may mean a lot of positive samples may be missed.

\subsection{F-measure/F-score}

This is defined as the harmonic mean of precision and recall in order to get one number as an average measure of performance.

$$ F_1 = \frac{2 \cdot precision \cdot recall}{precision + recall} $$

\subsection{Micro and Macro Averaging} \label{ssec:eval_metric_averaging}

Macro-averaging involves taking an average on the class level. Metrics are calculated for each class and then averaged at the end. Micro-averaging involves taking an average on the item level (i.e., taking the average of each of TP, FP, TN and FN to get the averages metrics).

\section{Baselines for Comparison}

In order to measure the performance of the system against current solutions it is useful to have a list of baseline performances. This way it can be inferred if there is an improvement being made by any newly created systems. Examples of systems that can be used include the reconstruction algorithm posed by H. Rebecq \textit{et al.}\cite{spikingToVideo} and existing gesture recognition algorithms for the DVS128 dataset\cite{DVS128} like the one posed by Arnon Amir \textit{et al.}\cite{eventBasedGestureRec} (See both in \Cref{sec:existing_algorithms}).

\section{Additional Testing with Camera}

If the model were to train using only the datasets available, there is a risk that the data would be unbalanced and the network is training on a very specific set of idealised readings. Testing with extraneous data generated by a camera under various different conditions would allow for evaluating the performance of the networks on data that is completely unseen and dissimilar.
% \chapter{Ethical, Legal and Safety Plan} \label{chap:ethical_plan}

In general the project in unproblematic, and poses minimal safety risks other than the ones presented when working with computers for extended periods of time (e.g., RSI, eye-strain, back aches etc.). In terms of ethical and legal considerations, however, it is important to make sure to use the practical camera setup in a sound manner. When collecting data from others it is important to be mindful of privacy issues. Consent must be taken from any participants if ever the need arises to take videos of others, but for the vast majority of project only videos of myself will be taken. As well as this it is important to consider where the system may be used if it is every put into production. Especially in IoT applications, privacy is of utmost concern, luckily with spiking data, if an intermediate video reconstruction cannot be obtained from the system, very little information can be discerned from spiking data. Therefore it would be very difficult to identify any individual within neuromorphic data itself.
\chapter{Analysis and Design} \label{chap:analysis_and_design}

\section{Requirements Capture}

% Projects with a deliverable that serves a specific 
% function often have an initial phase in which expected 
% use is investigated and a brief more detailed than the 
% specification is constructed. This would include what 
% is necessary, what is desirable, etc in the final 
% deliverable. The results of requirements capture 
% determine project objectives and are used to inform 
% project evaluation.
% Requirements capture is important in all projects with 
% real-world deliverables, and is often a significant 
% amount of work in software projects. Where 
% requirements capture is less relevant (for example in
% an analytical ‘research-style’ project) this may be 
% replaced by a detailed description of the project aims 
% and objectives in the Introduction or the Background 
% sections.

\section{Hardware and Software}

\subsection{Programming Languages}

When choosing a programming language for the project there were a few choices that are most often chosen by developers; Python\cite{Python}, R\cite{R}, and C++\cite{C++}. Python is the most popular choice due to the ease with which algorithms can be developed using it. Python features an extensive list of libraries and debugging functions that are invaluable when creating machine learning algorithms in particular, making the language of choice for this project. It is also important to note, however, the benefits of the other language options. C++ often results in programs with impressive performance due to the ability it grants to make low-level processes more efficient. Unfortunately this fine-level control also opens up programmers to more a demanding and time-intensive programming experience, with much more code writing and debugging to be done manually. R would also be a great choice for machine learning, and shares many similarities with Python, being open-source and having a huge community of developers constantly building libraries and tools. It has a different approach to machine learning, with a a more statistical analysis emphasis. Therefore Python remains the best choice for a more general approach for data processing.

\subsection{Machine Learning Frameworks and Software}

\subsubsection{PyTorch}

Pytorch\cite{Pytorch} is a relatively new framework for machine learning, and provides a developer friendly way to write machine learning code. It is a more `pythonic' approach to code abstraction that its competition in the space, and the \emph{torch.nn.module} gives access to clear, reusable module definitions in an Object-Oriented Programming manner. It also allows for simple data parallelism, so that batch processing can easily be split over different sets of hardware. It also has an intuitive debugging experience since itt can use standard debugging tools such as PyCharm and pdb.

\subsubsection{Tensorflow}

Tensorflow\cite{Tensorflow} is the older and more widely adopted machine learning library. It provides a more robust set of functionality with clear documentation. In terms of deployment it is the clear favourite as it allows models to be deployed on specialised servers and even on mobile. When visualising data software such as TensorBoard are ideal as it includes functionality to display model graphs, variables, histograms and much more. As well as this, Keras\cite{Keras} is a framework developed by Google, and uses a primarily Tensorflow based back-end. It provides an easy to use API for fast prototyping abd high levels of abstraction. It is used commercially by a plethora of companies and has a vast and highly developed research community. For these reasons Keras using a Tensorflow back-end were chosen for this project.

\subsection{Other Software}

\subsubsection{SpikingJelly}

SpikingJelly\cite{SpikingJelly} is an open-source deep learning framework for Spiking Neural Network (SNN) based on PyTorch. It allows for the processing of many often-used datasets in the neuromorphic community (Some of which are described in \cref{sec:existing_datasets}), as well as a simple set of classes for building SNNs or converting ANNs to SNNs. The library, which was mainly co-developed by Multimedia Learning Group, Institute of Digital Media (NELVT), Peking University and Peng Cheng Laboratory, can be installed directly via the \emph{pip} command. Using it, data can be loaded as event streams, as well as integrated frames (described in \cref{ssec:frame_integration}) of varying frame lengths or frame-rates. As well as this the package features clear documentation and tutorials to begin analysing neuromorphic data.

\subsubsection{NengoDL} \label{sssec:nengo}

NengoDL\cite{NengoDL} is a software framework designed to combine the strengths of neuromorphic modelling and deep learning. It is a useful tool for constructing biologically inspired spiking neuron models, and combining them to create fully spiking neural networks. As well as these these networks are intermixed with efficiently simulated deep learning concepts such as convolutional neural networks. This unified framework therefore allows us to train SNNs in the same way as we would for other ANN models with an easy to use interface. As well as this converters exist in order to convert ANNs to SNNs with relative ease.

\color{red} TODO: This could be moved if Nengo doesn't end up being used. \color{black}

\subsection{Cloud Environments}

Since Python is the language of choice for the project, Python Notebooks are a good choice for code segmentation and presentation. They allow for python code to be written in executable cells, so that their output as well as other text can be presented in a full document. This makes the code easy to understand for others, and good for development as well. Python notebooks can be run locally on a web server, as well as online on a cloud service. Many machine learning frameworks and algorithms make use of hardware acceleration using GPUs or TPUs. This means that code runs much faster on more powerful machines with this specialised hardware in them, which was not the case for hardware readily available during the course of the project. For this reason cloud services provided by the likes of Google and Amazon Web Services are a good alternative to physically owning hardware. They allow for the renting of GPUs etc. from their own servers, so that code can be run on them via their respective web services. 

The two main web services available at this time are Google Colaboratory\cite{GoogleColab} and AWS Sagemaker Studio Lab\cite{AwsSagemaker}. In terms of hardware, both services offer access to great GPUs, though sagemaker offers the more powerful T4 GPU at the free tier. This benefit, however, is not entirely relevant as a pro subscription would be necessary with either service to make use high-RAM runtimes. Due to Colaboratory's better share-ability and wider adoption, it was chosen as the service for this project.

\section{Datasets}

The datasets used for the evaluation of the networks in this project were NMNIST (\Cref{sssec:nmnist}), DVS128 Gesture (\Cref{sssec:dvs128_gesture}) and CIFAR10-DVS (\Cref{sssec:cifar10_dvs}), since these are the most suited for training on classification tasks. As well as this their recording conditions and topics are varied, meaning an evaluation of the system can be more robust and reliable. The event camera dataset for slam (\Cref{sssec:event_camera_dataset}) are also extensive and include side by side frames from a traditional camera, and so were ideal to use for testing the event reconstruction algorithms.

\section{End-to-end Event Classification Models}

\subsection{Data Pre-processing}

For classifying frames directly a frame integration method was utilised as described in \cref{ssec:frame_integration}. The method described involves having two channels per created frame. Each channel stores whether a positive or negative event was fired at any on pixel respectively. A slight adaptation of the classical method was also devised, in which rather than having binary information of on/off events per pixel of the frame, we could instead have a single channel value for each pixel with the sum of all event polarities for that location. This way the magnitude (or number) of events for each pixel can also be captured. A diagram showing the basic process with this custom integration method is shown in \cref{fig:frame_integration_diagram}.

\begin{figure}[htb]
    \centering
    \includegraphics[width=0.8\textwidth]{analysisanddesign/images/frame_integration_illustration.png}
    \caption{An illustration of the frame integration process with custom integrating method.}
    \label{fig:frame_integration_diagram}
\end{figure}

\color{red} TODO: Maybe write more about this and add more stuff about synchronous/asynchronous frame integration. \color{black}

\section{Two-phase Intensity Reconstruction Models}

In order to make use of existing highly-tested and documented computer vision techniques, intensity reconstruction models were first applied to events. For this reason a two-phase network (shown in \cref{fig:two_phase_network_pipeline}) was devised, wherein events are reconstructed into intensity videos before being passed into various classification networks..


\begin{figure}[htb]
    \centering
    \includegraphics[width=0.85\textwidth]{analysisanddesign/images/two_phase_network_pipeline.png}
    \caption{An illustration of the two-phase classification pipeline.}
    \label{fig:two_phase_network_pipeline}
\end{figure}

\color{red} TODO: Add system diagram of various networks being fed by E2VID and explain it. \color{black}

\section{Classification Networks}

\subsection{3-Dimensional Convolutional Neural Network} \label{ssec:conv_3d_network_design}

A 3D convolutional network as described in \cref{ssec:3D_conv_network} was implemented to work directly on the event data.

\subsection{Convolutional LSTM Network}

A convolutional LSTM network as described in \cref{ssec:conv_lstm} was implemented to work directly on the event data.

\subsection{Custom Convolutional LSTM Network}

This network is an extension of the LSTM network in the previous section, adding more capacity for the model to learn spatio-temporal patterns. It involves altering the intermediate convolutional neural network that analyses each frame in the video time series. A diagram showing the pipeline of the overall network can be seen in \cref{fig:custom_conv_lstm_pipeline}. This additional capacity is helpful for deciphering more complex `actions' in a video, such as in the case of gesture recognition.

\begin{figure}[htb]
    \centering
    \includegraphics[width=0.65\textwidth]{analysisanddesign/images/custom_conv_lstm_pipeline.png}
    \caption{An illustration of the custom convolutional LSTM network.}
    \label{fig:custom_conv_lstm_pipeline}
\end{figure}

\section{Evaluation Metrics} \label{sec:evalutaion_metrics}

The evaluation plan is as given by the typical machine learning pipeline\cite{IntroToML}. \color{red} TODO: Change this to actual used evaluation metrics and check it fits properly in background \color{black}

For a classification task, when we obtain the results from the test dataset (as shown in \cref{tab:possible_results}) we can calculate a variety of evaluation metrics that give various insights on our final model.

\begin{table}[htb]
    \centering
    \begin{tabular}{|| c  | c ||}
        \hline
        Labels     & Predictions \\
        \hline \hline
        1          & 1           \\
        \hline
        1          & 2           \\
        \hline
        3          & 8           \\
        \hline
        9          & 9           \\
        \hline
        6          & 9           \\
        \hline
        $ \vdots $ & $ \vdots $  \\
    \end{tabular}
    \caption{A table showing an example of results when inputting test data from NMNIST dataset\cite{NMNIST} into the final model.}
    \label{tab:possible_results}
\end{table}

\subsection{Confusion matrix}

Confusion matrices act as a visualisation of a systems performance. It shows possible true labels as well as possible predicted labels on either side, and filled in are the number of results that fit in each segment. In \cref{tab:confusion_matrix} the confusion matrix for the NMNIST dataset is shown as an example. It should be noted that a similar confusion matrix should be created taking each class as positive, then each metric can be calculated by taking the averages (as shown in \cref{ssec:eval_metric_averaging}). For each of the cells the number of matching records are stored to calculate each of the evaluation metrics. The table includes True Positives (TP), False Positives (FP), True Negatives (TN) and False Negatives (FN).

\begin{table}[htb]
    \centering
    \begin{tabular}{|| c c | c | c | c | c ||}
        \hline
                                                                         &                                    & \multicolumn{4}{ c ||}{\textbf{Predicted Class}}                                        \\
        \cline{3-6}
                                                                         &                                    & 1                                                & 2          & 3          & $ \hdots $ \\
        \hline
        \multirow{6}{*}{\rotatebox[origin=c]{90}{\textbf{Actual Class}}} & \multicolumn{1}{| c |}{1}          & TP                                               & FN         & FN         & $ \hdots $ \\
        \cline{2-6}
                                                                         & \multicolumn{1}{| c |}{2}          & FP                                               & TN         & TN         & $ \hdots $ \\
        \cline{2-6}
                                                                         & \multicolumn{1}{| c |}{3}          & FP                                               & TN         & TN         & $ \hdots $ \\
        \cline{2-6}
                                                                         & \multicolumn{1}{| c |}{4}          & FP                                               & TN         & TN         & $ \hdots $ \\
        \cline{2-6}
                                                                         & \multicolumn{1}{| c |}{5}          & FP                                               & TN         & TN         & $ \hdots $ \\
        \cline{2-6}
                                                                         & \multicolumn{1}{| c |}{$ \vdots $} & $ \vdots $                                       & $ \vdots $ & $ \vdots $ & $ \ddots $ \\
    \end{tabular}
    \caption{a table showing one particular confusion matrix for NMNIST dataset\cite{NMNIST} for class 1 as the positive class.}
    \label{tab:confusion_matrix}
\end{table}

\subsection{Accuracy}

The accuracy of the system is the proportion of samples correctly classified.

$$ Accuracy = \frac{TP + TN}{TP + TN + FP + FN} $$

Note: classification error can also be used and is defined as $ 1 - accuracy $.

\subsection{Precision}

Precision is the proportion of positively predicted samples identified correctly.

$$ Precision = \frac{TP}{TP + FP} $$

It should be noted that a high precision may mean that there are many false positives.

\subsection{Recall}

Recall is the proportion of actual positives correctly classified.

$$ Recall = \frac{TP}{TP + FN} $$

It should be noted that a high recall may mean a lot of positive samples may be missed.

\subsection{F-measure/F-score}

This is defined as the harmonic mean of precision and recall in order to get one number as an average measure of performance.

$$ F_1 = \frac{2 \cdot precision \cdot recall}{precision + recall} $$

\subsection{Micro and Macro Averaging} \label{ssec:eval_metric_averaging}

Macro-averaging involves taking an average on the class level. Metrics are calculated for each class and then averaged at the end. Micro-averaging involves taking an average on the item level (i.e., taking the average of each of TP, FP, TN and FN to get the averages metrics).
\chapter{Implementation} \label{chap:implementation}

This chapter presents the final implementation of the project. The full process for both the end-to-end event classification models, as well as the two-phase video reconstruction networks are given, alongside any data pre-processing that was required.

\section{Data Preprocessing}

As described in  \cref{ssec:data_preprocessing_design}, data had to be centered to give a z-score. In practice the standard deviation was not used to normalise the values, and instead the following operation was undertaken: $ \textbf{x} = \frac{\textbf{x} - mean_x}{max_x} $. However, during the implementation of the networks it was evident that the default Keras functions would not be ideal. The reason for this is that the dataset, as real-world data often is, were far too large to store on the RAM of a computer. This is the case even with the high-RAM run-times of Google Coloboratory. This meant that data had to be read into the network on a per-batch basis, making the whole process less RAM intensive. Keras has a \lstinline{Sequence} class that can be extended to create a custom data-generator class. \color{red} TODO: Maybe put code example here. \color{black}

\section{Two-phase Intensity Reconstruction Pipeline}

The models described above allow for the system to learn directly on the spiking data. Another approach that was taken was to attempt to utilise video reconstruction networks to the event data so that more classical models and architectures could be used to patterns in the data.

\subsection{Reconstruction Algorithms}

\subsubsection{E2VID}

E2VID, as described in \cref{ssec:video_reconstruction}, is a state-of the art network based on UNET that reconstructs intensity videos from events data. Having gotten the output from the network (which on test data had an average Mean-Squared Error (MSE) of 0.05), it now needed to be processed slightly in order to work with the classification models. The main issue with the reconstruction was that the video were of varying lengths. In order to mitigate this additional frames were added to videos which had fewer frames than that of the longest video in the training set. These additional frames were added by repeating the video until the desired number of frames was reached.

\color{black}

\section{End-to-end Event Classification Pipeline}

\subsubsection{Frame Integration}

In order to begin analysing neuromorphic data, it was pre-processed it into a form that a NN can take as input. For most the testing an implementation of the more formal method of processing events streams to get integrated frames given in \cref{ssec:frame_integration} was implemented. This technique was also applied to another readily available data-sets, the DVS128 Gesture dataset.

The integrated frames of a hand gesture in \cref{fig:dvs128_integrated_frames} shows the motion of an arm moving in a clockwise direction \color{red} TODO: Show classic frame integration for NMNIST dataset \color{black}. For each instance of a gesture the event stream was split into 20 frames. This made the processing of the data easy for some of the networks described later in the chapter, since the frames could be packed into 3D tensors of equal size. This means that the frames are not synchronous like they would be from an frame-based camera, and the amount of time represented by each frame is varying. As well as this, since the event streams are of varying length in the time dimension, some videos are reconstructed to a more granular scale than others. It is apparent that in this particular sample a relatively large amount of time is being compressed into every frame, resulting in a large amount of motion being visible. With more frames being created for each sample this problem would be alleviated and more easily distinguishable features may be seen. However if too many frames are taken, not only are processing times greatly increased, events may be sparse and not show any visible pattern in any one frame. A benefit of having this blurring event in the intensity frames is that the direction and degree of motion can be seen in the frames, and so the networks can also pick up these patterns in their classification process.

\begin{figure}[htb]%
    \centering
    \subfloat[\centering]{{\includegraphics[width=0.25\textwidth]{implementation/images/Dvs128_integrated_frame_1.png}}}%
    \qquad
    \subfloat[\centering]{{\includegraphics[width=0.25\textwidth]{implementation/images/Dvs128_integrated_frame_2.png}}}%
    \qquad
    \subfloat[\centering]{{\includegraphics[width=0.25\textwidth]{implementation/images/Dvs128_integrated_frame_3.png}}}%
    \caption{Three contiguous intensity frames \textbf{(a)}, \textbf{(b)} and \textbf{(c)}, created from the DVS128 Gesture dataset with a person moving their right hand clockwise.}%
    \label{fig:dvs128_integrated_frames}%
\end{figure}

\color{red} TODO: Add more photos of integrated frames with more and less frames \color{black}

\color{red} TODO: Add pseudocode for frame integration \color{black}

\subsubsection{Custom Frame-Integration}

As well as this a novel integration technique was implemented, as described in the \cref{sec:end_to_end_classification_design}. The method involved segmenting the events into groups based on their timestamp. \Cref{fig:nmnist_spikes_to_intensity_map} shows a visualisation of intensity maps created from the NMNIST\cite{NMNIST} dataset \color{red} TODO: Show custom frame integration of DVS128 Dataset \color{black}. The set of all events was split into eight segments, where each segment included events within a range of $ 1 \times 10^6 $ ms (i.e., $ 0 \rightarrow 1 $, $ 1 \rightarrow 2 $, ..., $ 7 \rightarrow 8 $). This way the data representation shifted to somewhat get back to a set of frames that mimicked the video output usually seen from everyday cameras. \textbf{(a)} shows the segmented events visualised in three dimensions (x\_location, y\_location and timestamp). In \textbf{(b)} these events were projected onto the two dimensional plane (of x\_location and y\_location), then the plots for on events and off events are shown separately. Finally in \textbf{(c)} an intensity map was created from the projected events. Each pixel in the intensity map grid was initialised to 0, and for every on event 1 was added to the cell, and for every off event 1 was subtracted from the cell. This way temporal information was retained to a greater degree than if each pixel simply had binary information of whether a event occurred or not in a two channel image (as was done by the spikingjelly package\cite{SpikingJelly}). It was clear that the resulting output greatly resembled the MNIST\cite{MNIST} sample recorded by the ATIS camera (As shown in \cref{fig:nmnist_spikes_visualisation} in \cref{sec:existing_datasets}). Another method that aimed to preserve temporal resolution was proposed by Lo\"ic Cordone \textit{et al.}, who proposed `voxel cubes'\cite{MiniVovelCubes}. For this method there were still binary events in every channel, but there were more than two channels so that the events in each time-slice could be subdivided into each channel. When compared to this method, the temporal information can be stored in the same way with the new proposed method, while keeping data size small since it is just a one-channel image. 

\begin{figure}[htb]%
    \centering
    \subfloat[\centering]{{\includegraphics[width=0.25\textwidth, height=0.7\textwidth]{implementation/images/nmnist_spikes_visualisation_segmented.png}}}%
    \qquad
    \subfloat[\centering]{{\includegraphics[width=0.4\textwidth, height=0.7\textwidth]{implementation/images/nmnist_events_segmented.png}}}%
    \qquad
    \subfloat[\centering]{{\includegraphics[width=0.25\textwidth, height=0.7\textwidth]{implementation/images/nmnist_events_heatmap_segmented.png}}}%
    \caption{A visualisation of intensity maps created by segmenting events into bins of size $ 1 \times 10^6 $ ms.}%
    \label{fig:nmnist_spikes_to_intensity_map}%
\end{figure}

\color{red} TODO: Add stuff about possible nlp-like networks. \color{black}

\section{Classification Models}

The models used to classify the either the integrated frames or video reconstructions are given below. Please note that the input shapes given below are for the NMNIST dataset, however for the other datasets different input shapes would be required (e.g. {128, 128, 2} for the DVS128 Gesture dataset).

\subsection{3D Convolutional Neural Network}

The 3D convolutional neural network in \cref{ssec:3D_conv_network} was altered to have the outputs of the hidden layer fed into a dense layer with 10 neurons to classify the correct class (0-9). Activation functions need to be present in the network to prevent all the layers from becoming equivalent to a single one (linear regression model). In order to learn more complex patterns activation functions are a necessary aspect of creating an artificial neuron (See \cref{eq:artificial_neuron_output} in \cref{ssec:snn_and_heterogeneity}). The most commonly used activation function in deep networks (and image recognition in particular) is ReLU, so that was the natural choice for this network as well. Finally, the output layer has a sigmoid activation function. This function compresses the output smoothly between the ranges of 0 and 1. this means each of the outputs from the neurons can be interpreted as a the probability of the input being any one of the 10 classes. This means we can simply take the highest probability as the predicted class from the network.

In order to choose the most appropriate parameters for the system, as well as the other systems implemented during the course of this project, multiple tests were run varying each of the possible hyper-parameters. The tests conducted were to determine;

\begin{itemize}
    \item The number of hidden layers.
    \item The number of neurons per layer.
    \item The size of convolution kernels.
    \item The introduction of some fully connected layers after convolutional layers.
\end{itemize}

The network in each case was trained for multiple epochs. The performance of a typical network can be seen in \cref{fig:accuracy_and_loss_per_epoch}, where the performance on the training set steadily improves as the network progresses through epochs. Accuracy is one of the metrics defined in \cref{sec:evalutaion_metrics}, and the loss for the given network is called categorical cross-entropy loss. The formula used to calculate the loss is given by: $ L = -\frac{1}{N}\sum^N_{i=1}\sum^C_{c=1}y_c^{(i)}log(\hat{y}_c^{(i)}) $, where there are $ N $ samples and $ C $ classes. $ y_c^{(i)} $ is $ 1 $ when the class is correctly predicted and $ 0 $ otherwise and $ \hat{y}_c^{(i)} $ is the predicted probability of class $ c $ for data-point $ i $. This is the same loss function that was used in the other networks in the report as well.

\begin{figure}[htb]
    \centering
    \includegraphics[width=0.4\textwidth]{implementation/images/accuracy_and_loss_per_epoch.png}
    \caption{A figure showing classification accuracy and cross-entropy loss per epoch on training data for a typical network.}
    \label{fig:accuracy_and_loss_per_epoch}
\end{figure}

It was clear, however, that these results may be misleading since they only represent the efficiency of the system when classifying values within the training data-set. However, when looking at the performance on an unseen test data-set, it is obvious that some of the features learnt do not easily translate to general trends in unseen data. This is known as over-fitting, and can be avoided by reducing the capacity of the data-set so that it does not learn information specific to the training set, or by stopping the training process earlier.

Model capacity is directly correlated to the n$ ^o $ of filters, as well as the number of layers, and as the model recognises more patterns in the training data, so it is important to get the optimum value for the system. The size of the kernels has an effect on the scale of the information picked up by the system. With smaller kernels more local patterns are detected, whereas with larger kernels more global effects can be seen. An example of the effects of changing such hyper-parameters can be seen in \cref{fig:hyperparameter_tests} As for the dense layers at the end of the network, it can be seen that better results were achieved as a result of it since global patterns can be further identified after the data has been processed by the convolution layers that have picked out the most important features.

\begin{figure}[htb]%
    \centering
    \subfloat[\centering]{{\includegraphics[width=0.4\textwidth]{implementation/images/layer_tests.png}}}%
    \subfloat[\centering]{{\includegraphics[width=0.4\textwidth]{implementation/images/kernel_tests.png}}}%
    \caption{Graphs showing the effect on training loss with varying hyper-parameters.}%
    \label{fig:hyperparameter_tests}%
\end{figure}

Further testing was done with increasing kernel sizes, and with pooling layers added to the network. 

An overview of the layers present in the network, together with the shape of their outputs and number of trainable parameters can be seen in \cref{lst:3d_conv_layers}. It features the repeating pattern of 3D convolution layers and max-pooling. The number of filters in each layer keeps increasing (from 32 to 64 and 128). The function of 2D filters is to capture patterns in the data, and as you move forward through the network these patterns get more complex. For example, if dots and lines are captured in the first layer, shapes such as triangles and squares may be captured in the second one. Since the patterns are more complex, the number of possible combinations also increases. This is the reason more filters are required for later layers. The max-pooling layers scale down the image, this also means that more large scale patterns can be identified from smaller sections of the frame.

\subsection{Convolutional LTSM}

An overview of the layers present in the network, together with the shape of their outputs and number of trainable parameters can be seen in \cref{lst:conv_lstm_layers}. The thought behind the structure of this network is similar to the one when designing the 3D convolutional network. The number of filters increases as you go further through the network. The difference this time is that 2D convolution is carried out on each contiguous frame of the video input before being passed through LTSM networks. The way this is implemented is with the convLTSM network described in \cref{ssec:conv_lstm}.

\subsection{Custom Convolutional LSTM}

An implementation of a custom convolutional LSTM can be seen in \cref{lst:custom_conv_lstm_layers} and the implementation of the custom 2D convolutional network applied to each frame of the video can be seen in \cref{lst:custom_conv_2d_layers}. This network is the natural progression from the previous convLSTM network since it applies a more complex 2D convolution to each frame. This extra capacity is evident in the number of trainable parameters in the new network when compared to the one in \cref{lst:conv_lstm_layers}.

\subsection{Spiking Neural Network}

For the basis of the spiking neural network, the same 2D convolutional neural network used in the custom convolutional LSTM network (See \cref{lst:custom_conv_2d_layers}). First an ANN was constructed and trained with traditional back-propagation. Since there were a series of frames being input to the network, the actual labels had to be repeated for each frame during the training process. Once this training was complete, the network was converted to a spiking neural network. For this the activation of the neurons in the network were changed from tensorflow's \lstinline{nn.relu} to nengo's \lstinline{SpikingRectifiedLinear}. The specifics for the underlying conversion can be seen in the work done by Bodo Rueckauer \textit{et al}\cite{Ann2Snn}. The neural activity of this network can be seen in \cref{fig:pre_snn_conversion}. The input to this network was a integrated frame video with each frame repeated 8 times (as described in \cref{ssec:snn_design}). The plots are shown over time, and since our network doesn't have any temporal elements (i.e. spiking neurons), the neural activity is for the whole duration it is run. Since the neuron activations are continuous theyH each have a constant rate of neural activity. For this exploratory work the network was trained for 15 epochs, giving an approximate accuracy of 95.62\%.

\begin{figure}[htb]%
    \centering
    \includegraphics[width=0.6\textwidth]{implementation/images/pre_snn_conversion.png}
    \caption{Graphs showing neural activity of ANN before conversion to SNN.}%
    \label{fig:pre_snn_conversion}%
\end{figure}

Then the network was converted to a spiking neural network by changing the Rectified Linear (ReLU) layers to Spiking ReLU layers. As previously explained, each 3 dimensional vector needed to be sent to the network multiple times to allow for spikes to accumulate in the neural network. It was found that to achieve a good number of spikes in the network the frames had to be repeated 30 times. A graph showing the activity of neurons in the first convolutional layer of the network with high and low number of repetitions can be seen in \cref{fig:frame_repititions}. Graphs showing the performance of the converted network can be seen in \cref{fig:post_snn_conversion}. Note that for this system since data it being passed into it consecutively, the only the final output from the system is sampled for predictions. The accuracy of the converted network fell to approximately 10\%.

\begin{figure}[htb]%
    \centering
    \includegraphics[width=0.6\textwidth]{implementation/images/post_snn_conversion.png}
    \caption{Graphs showing neural activity of ANN after conversion to SNN.}%
    \label{fig:post_snn_conversion}%
\end{figure}

Once the conversion from ANN to SNN was complete, the performance of the network was incomparable to before the process. The reason for this (as outlined in \cref{ssec:snn_and_heterogeneity}) is the poor approximations that take place in the conversion process. In order to mitigate this, some network alterations were made to improve the approximations.

\begin{figure}[htb]%
    \centering
    \subfloat[\centering]{{\includegraphics[width=0.4\textwidth]{implementation/images/low_frame_repitition.png}}}%
    \subfloat[\centering]{{\includegraphics[width=0.4\textwidth]{implementation/images/high_frame_repitition.png}}}%
    \caption{Comparison of number of spikes with less \textbf{(a)} and more \textbf{(b)} repetitions.}%
    \label{fig:frame_repititions}%
\end{figure}

\subsubsection{Synaptic Smoothing}

The plots for the spiking neural network show that since the neuron function is now no-longer continuous, discrete spikes occur to propagate information throughout the network. This means that the output of the network is very noisy. Since there are so many neurons spiking at any one given time, there is no guarantee that when you sample the output at the final time-step that the correct neuron will be firing (even if it has a relatively higher firing rate). For this reason we need to apply a smoothing function to the firing neurons. Essentially the \lstinline{synapse} parameter in nengo applies a moving average filter to the network output so that the predictions are more stable. The effect of applying this filter can be seen in \cref{fig:post_synaptic_smoothing}. It was found that the performance of the system was best with smoothing=0.006.

\begin{figure}[htb]%
    \centering
    \subfloat[\centering]{{\includegraphics[width=0.4\textwidth]{implementation/images/post_synaptic_smoothing_1.png}}}%
    \qquad
    \subfloat[\centering]{{\includegraphics[width=0.4\textwidth]{implementation/images/post_synaptic_smoothing_2.png}}}%
    \qquad
    \subfloat[\centering]{{\includegraphics[width=0.4\textwidth]{implementation/images/post_synaptic_smoothing_3.png}}}%
    \caption{Performance of spiking neural network with progressively more synaptic smoothing; \textbf{(a)} smoothing=0.002 \textbf{(b)} smoothing=0.005, and \textbf{(c)} smoothing=0.010.}%
    \label{fig:post_synaptic_smoothing}%
\end{figure}

\subsubsection{Post-training Firing Rate Scaling}

Neuron updates only occur when it outputs, or 'fires' a spike. This means that the system is much more likely to update when firing rates increase. Since the only output function that has been replaces is ReLU, which is a linear function, it is feasible to multiply the inputs of the neurons by a scale factor, then divide the output by the same number to scale. Larger inputs mean the neuron is more likely to spike, yet the linearity of the function is maintained by scaling down the output. The effect of applying this filter can be seen in \cref{fig:post_firing_rate_scaling}. It is important to note, however, that if the firing rate were to theoretically be scales infinitely, the neuron functions for be continuous, exactly replicating artificial neural networks. This may be good in terms of efficiency, but negates the point of using spiking neural networks in the first place. It was found that the best compromise for this system was to scale the firing rate by 150, which preserved the benefits of the spiking neural network and also improved accuracy back to the levels before the conversion.

It is also possible to change the effective firing rates of each neuron in the network, we could also add a custom loss function that incentivises a certain firing rate during the training processes as well. The benefit of this is that it can be applied to non-linear functions other than ReLU, but since there is no such function present in this project it need not be implemented.

\begin{figure}[htb]%
    \centering
    \subfloat[\centering]{{\includegraphics[width=0.4\textwidth]{implementation/images/post_firing_rate_scaling_1.png}}}%
    \qquad
    \subfloat[\centering]{{\includegraphics[width=0.4\textwidth]{implementation/images/post_firing_rate_scaling_2.png}}}%
    \qquad
    \subfloat[\centering]{{\includegraphics[width=0.4\textwidth]{implementation/images/post_firing_rate_scaling_3.png}}}%
    \caption{Performance of spiking neural network with progressively more firing rate scaling; \textbf{(a)} scaling=20 \textbf{(b)} scaling=50, and \textbf{(c)} scaling=100.}%
    \label{fig:post_firing_rate_scaling}%
\end{figure}
\chapter{Testing and Results} \label{chap:testing_and_results}

In order to evaluate each of the networks implemented in the previous chapter, various tests were conducted. The main forms of evaluation used are outlined in \cref{sec:evalutaion_metrics}. These metrics allow for a comprehensive understanding of network performance. As well as this some explanation of the findings are also given.

\section{Data Pre-processing}

The data was converted to the z-score as described in \cref{ssec:data_preprocessing_design}, which made the network learn more stably and ensure any patterns found were more robust and easily applicable to unseen data. This effect was much more apparent on the reconstructed video sequences since the intensities tended to vary much more than the intensities from integrated frames. This is since the values from the integrated frames were naturally `centred' around similar values.

\section{Two-phase Intensity Reconstruction Models}

Most current classification networks are built to harness the features of video streams from frame-based cameras. To this end, networks were created to use these networks on intensity reconstructions from pre-built networks like E2VID (as described in \cref{ssec:video_reconstruction}). The results of such a network are given in this next section.

\subsection{Intensity Reconstruction}

The E2VID reconstruction network was used to recreate intensity videos from events. It was evident that the reconstructions created were robust and relatively to to life. The reconstructions were not effected by adverse lighting effects or fast motions, which can be verified in \cref{fig:intensity_reconstructed_frames}, which shows a reconstruction of a video of a runner in a sunny environment. It is evident that even in sunny conditions the event camera is able to capture high contrast details such as buildings in the background against the sunlight. Now, modern computer vision techniques could still be applied to event data, while still preserving the many benefits the event model presents. It was interesting to note the performance of the reconstruction model on inputs with few moving parts. Since events are only triggered when there is an intensity change on any given pixel on the sensor, only regions with motion in them showed up as events, and the nature of all the pixels with no motion was not easily inferable. This can clearly be seen in \cref{fig:wave_in_lightings_reconstructions}, where only parts of the scene were accurately reconstructed. This did not, however, pose much of a problem for tasks such as gesture recognition, since the motion is exactly what is being classified, however for other tasks such as object recognition it had to be ensured that there was some sort of motion of either the object or the camera for the reconstruction algorithm to be effective.

\begin{figure}[htb]%
    \centering
    \subfloat[\centering]{{\includegraphics[width=0.25\textwidth]{testingandresults/images/outdoor_frames.png}}}%
    \qquad
    \subfloat[\centering]{{\includegraphics[width=0.25\textwidth]{testingandresults/images/outdoor_recontstructed_frames.png}}}%
    \caption{Figure showing matching video \textbf{(a)} and reconstructed frames \textbf{(b)} from the event camera dataset\cite{EventCameraDataset}.}%
    \label{fig:intensity_reconstructed_frames}%
\end{figure}

\Cref{fig:wave_in_lightings_reconstructions} also shows that event-cameras do indeed allow for higher fidelity video capture in a wider range of lighting conditions that frame-based cameras (as explained in \cref{ssec:event_camera_benefits}). In all lighting conditions the video reconstruction was largely the same, since the events triggered were very similar in all cases. The logarithmic characteristics of the event-sensor pixels are the reason for this, since the thresholds for the triggering of events is not static.Because the reconstructions are consistent across lighting conditions, this also means that the reconstruction is more reliable, and the classification algorithm works better in adverse conditions in general \color{red} TODO: get figures and images to prove this \color{black}. The E2VID reconstruction uses fixed-size event windows. This means that each input has $ n $ events, and therefore has an equal amount of information. This way the output data can be of a high fidelity even with large amounts of motion. For example event though the motion of the hand is very large in this video, the individual fingers and details are still clearly visible in every frame.

\begin{figure}[htb]%
    \centering
    \subfloat[\centering]{{\includegraphics[width=0.18\textwidth]{testingandresults/images/dvs_wave_fluorescent_led.png}}}%
    \qquad
    \subfloat[\centering]{{\includegraphics[width=0.18\textwidth]{testingandresults/images/dvs_wave_fluorescent.png}}}%
    \qquad
    \subfloat[\centering]{{\includegraphics[width=0.18\textwidth]{testingandresults/images/dvs_wave_lab.png}}}%
    \qquad
    \subfloat[\centering]{{\includegraphics[width=0.18\textwidth]{testingandresults/images/dvs_wave_led.png}}}%
    \qquad
    \subfloat[\centering]{{\includegraphics[width=0.18\textwidth]{testingandresults/images/dvs_wave_natural.png}}}%
    \caption{A waving motion being reconstructed from events captures by DVS128 event camera under different lighting conditions.The lighting conditions are as follows; \textbf{(a)} fluorescent led, \textbf{(b)} fluorescent, \textbf{(c)} lab lighting, \textbf{(d)} led lighting and \textbf{(e)} natural lighting.}%
    \label{fig:wave_in_lightings_reconstructions}%
\end{figure}

However, the quality of the reconstructions was not as impressive for datasets where the input is vastly different to the data E2VID was trained on (see \cref{fig:nmnist_reconstructions}). For example, with the NMNIST dataset\cite{NMNIST}, the input size was very small. This was a deliberate choice so that the format of NMNIST closely matches the format of MNIST, allowing for efficient training and testing of networks. However, the reconstruction of these samples was very rudimental, showing results similar to an edge map rather than a full reconstruction. This is one of the factors that made the two-phase pipeline less effective for this dataset as compared to the DVS128 Gesture reconstructions. This disparity could be eliminated if E2VID was re-trained on the specific data that was input, however for this project such datasets were not readily available, nor the time to train the network.

\begin{figure}[htb]%
    \centering
    \includegraphics[width=0.6\textwidth]{testingandresults/images/nmnist_reconstructions.png}
    \caption{A figure showing three reconstructed frames from a recording of the MNIST character `6'.}%
    \label{fig:nmnist_reconstructions}%
\end{figure}

\subsubsection{Classification Results}

The confusion matrices for the classification of the intensity reconstructed NMNIST dataset can be seen in \cref{fig:nmnist_recon_c_matrices}. It is clear that the performance of the network is worse on more challenging characters such as 6, 8 and 9. This is understandable, considering the corresponding reconstructions of these classes is of noticeable poorer quality (see \cref{fig:nmnist_reconstructions}). Although the reconstructions are sometimes distinguishable by eye, the lines are often blurred. Due to the small scale of the dataset the resulting video stream is also of a small size, meaning less details can could be represented in the relatively low resolution images.

For the 3D convolutional neural network, the precision was often lower for these difficult classes. This means that these classes were over-represented in the predictions, when it should have been the other classes being detected (i.e., false positives). The purely convolutional network was able to identify patterns in the data due to its large number of trainable parameters, however this also meant that it over-trained on the features present in the more complex classes, often finding them in samples of the other classes.

Interestingly, with the LSTM networks it is instead the recall that was lower for these challenging classes, meaning many of the samples of these classes were misclassified (false negatives). The cause for this may be that the had fewer trainable parameters, but the recurrent capabilities made up for this allowing for the network to learn temporal patterns more reliably.

Overall the f1 measure acts as a balanced measure between precision and recall, since the data is not biased to any one class. The f1 score is superior for the 3D convolutional network and custom convolutional network, which both have similar performance on the test dataset. This makes sense since the temporal patterns are less important in the case of number classification, since motion is not prevalent or interesting.

\begin{figure}[htb]%
    \centering
    \subfloat[\centering]{{\includegraphics[width=0.4\textwidth]{testingandresults/images/c_matrix_nmnist_recon_conv3d.png}}}%
    \qquad
    \subfloat[\centering]{{\includegraphics[width=0.4\textwidth]{testingandresults/images/c_matrix_nmnist_recon_convlstm.png}}}%
    \qquad
    \subfloat[\centering]{{\includegraphics[width=0.4\textwidth]{testingandresults/images/c_matrix_nmnist_recon_custom_convlstm.png}}}%
    \caption{Confusion matrices for intensity reconstructed NMNIST classification with various networks; \textbf{(a)} conv3D, \textbf{(b)} convLSTM, \textbf{(c)} custom convLSTM.}%
    \label{fig:nmnist_recon_c_matrices}%
\end{figure}

\Cref{tab:conv3d_nmnist_recon_evaluation_metrics}, \cref{tab:conv_lstm_nmnist_recon_evaluation_metrics}, and \cref{tab:custom_conv_lstm_nmnist_recon_evaluation_metrics} show the performance evaluation of each of the networks on the intensity reconstructed NMNIST dataset in more detail.

\vspace{10pt}

The confusion matrices for the classification of the intensity reconstructed DVS128 Gesture dataset can be seen in \cref{fig:dvs128_recon_c_matrices}. Here the findings were a little different than for the NMNIST dataset. Firstly, it is important to note that when attempting to reconstruct the videos and pass them into each of the networks, the amount of memory often became an issue. This in itself is a clear indication of a large difference between the event and frame representations. After conversion, the frame-based video was of a much larger size and density than the event streams it was created from. In other words the efficiency of encoding was much higher in the case of event streams than intensity videos. For this reason it was not even feasible to run the ConvLSTM network with this data since both the memory and time requirements were much too high.

Between the purely convolutional and LSTM network there were still many interesting conclusions to be drawn. Th performance disparity between the two was of a much larger scale than with the reconstructed NMNIST dataset. The reason for this is that the recurrent neural network was much better at recognising temporal patters as well as spatial ones in the data, which is much more important in the case of gesture recognition. For this reason the overall accuracy was much better with the LSTM network. For cases with very similar frames (such as clockwise and counter-clockwise arm rotations), the LSTM had far better precision as well.

\begin{figure}[htb]%
    \centering
    \subfloat[\centering]{{\includegraphics[width=0.4\textwidth]{testingandresults/images/c_matrix_dvs128_recon_conv3d.png}}}%
    \qquad
    \subfloat[\centering]{{\includegraphics[width=0.4\textwidth]{testingandresults/images/c_matrix_dvs128_recon_custom_convlstm.png}}}%
    \caption{Confusion matrices for intensity reconstructed  DVS128 Gesure classification with various networks; \textbf{(a)} conv3D, \textbf{(b)} custom convLSTM.}%
    \label{fig:dvs128_recon_c_matrices}%
\end{figure}

\Cref{tab:conv3d_dvs128_recon_evaluation_metrics} and \cref{tab:custom_conv_lstm_dvs128_recon_evaluation_metrics} show the performance evaluation of each of the networks on the intensity reconstructed DVS128 Gesture dataset in more detail. 

\section{End-to-end Event Classification Models}

This section shows the results obtained from the end-to-end event classification pipeline. The benefits of the event driven camera (as described in \cref{ssec:event_camera_benefits}) were evident in the acquired results. As opposed to tradition frame-based cameras, high frequency data is not lost when processing events. There are spikes for every event at a much more granular scale in the temporal dimension in the event camera when compared to the frame camera, meaning fast movements were captured more reliably since events are captured at the $\mu s$ scale, no longer restricted by the frame-rates of modern cameras (which often results in motion blur).

\subsection{Frame Integration}

It is evident that modern computer vision techniques have been developed with frame-based cameras in mind, and so modern networks achieve good accuracy, and are able to find patterns well, on frame-based data. For this reason the common technique of integrating frames \cref{ssec:frame_integration} results in frames, akin to the ones a regular camera generates. It does, however, still pose many benefits when compared to the frames from a regular camera. As mentioned previously information between frames is still not lost or degraded since the events are still captured and visible in each frame. As well as this, the frames generated from events inherently focussed on the points of interest in the image, since these were the only ones in motion in the frame. Most common architectures (such as the one proposed by Raimundo F. Pinto \textit{et al.} for static hand gesture recognition\cite{StaticHandGesture}) feature an intermediate layer to remove backgrounds and other noise from images to focus on the points of interest. With frame integration, this intermediate layer could be omitted, since the output was already similar to an edge map. It is conceivable, however, that in noisy environments with lots of motion this stage would still be necessary.

As previously mentioned, when using the frame integration method the result was very similar to an edge map, which is often the primary step of image analysis using convolutional neural networks to images in existing networks already. \Cref{fig:canny_edge_detection_nmnist} shows the effect of carrying out canny edge detection\cite{CannyEdgeDetection} on an integrated frame from the NMNIST dataset. The sample is taken from a recording of the number `0', and it can be seen that the original frame is in essence just a noisy edge map of the number. The steps taken for canny edge detection were as follows;

\begin{enumerate}
    \item Smooth image by convolving with an averaging filter of the form: $ \begin{bmatrix}
        \frac{1}{4^2} & \frac{1}{4^2}  & \frac{1}{4^2}  & \frac{1}{4^2} \\
        \frac{1}{4^2} & \frac{1}{4^2}  & \frac{1}{4^2}  & \frac{1}{4^2} \\
        \frac{1}{4^2} & \frac{1}{4^2}  & \frac{1}{4^2}  & \frac{1}{4^2} \\
        \frac{1}{4^2} & \frac{1}{4^2}  & \frac{1}{4^2}  & \frac{1}{4^2} \\
    \end{bmatrix} $
    \item Sobel edge detection was carried out in order to find the edges of the smoothed image. To do this the image was convolved with the following matrices: $ S_x =
    \begin{bmatrix}
        -1 & 0 & 1 \\
        -2 & 0 & 2 \\
        -1 & 0 & 1 \\
    \end{bmatrix} $ and $ S_y = 
    \begin{bmatrix}
        1 & 2 & 1 \\
        0 & 0 & 0 \\
        -1 & -2 & -1 \\
    \end{bmatrix} $. These matrices got the pixel gradients in the x and y directions, and taking their magnitudes ( $ \sqrt{S_x^2 + S_y^2} $ ) gives the edges map of the frame.
    \item Hysteresis thresholding allowed for weaker edges to be counted as long as they are connected to stronger ones. Then non-maximum suppression was applied to get a single line for every edge (by finding the strongest point of every line in the direction of its gradient).
\end{enumerate}

This refined edge map could have been part of the pre-processing of the data before being passed into the network, but it was found that this was not beneficial to network training. This was because some information is lost in the smoothing process, and any feature mapping was done efficiently during the training of convolutional networks anyway.

\begin{figure}[htb]%
    \centering
    \subfloat[\centering]{{\includegraphics[width=0.18\textwidth]{testingandresults/images/denoise_origional.png}}}%
    \qquad
    \subfloat[\centering]{{\includegraphics[width=0.18\textwidth]{testingandresults/images/denoise_smoothed.png}}}%
    \qquad
    \subfloat[\centering]{{\includegraphics[width=0.18\textwidth]{testingandresults/images/denoise_edge_map.png}}}%
    \qquad
    \subfloat[\centering]{{\includegraphics[width=0.18\textwidth]{testingandresults/images/denoise_hysteresis.png}}}%
    \caption{Progression of canny edge detection on integrated frame of a sample from the NMNIST dataset with class `0'. The steps are as follows; \textbf{(a)} original integrated frame, \textbf{(b)} smoothed, \textbf{(c)} edge detection, and \textbf{(d)} Non-maximum suppression and hysteresis thresholding.}%
    \label{fig:canny_edge_detection_nmnist}%
\end{figure}

\subsubsection{Classification Results}

The confusion matrices for the classification of the frame-integrated  NMNIST dataset can be seen in \cref{fig:nmnist_c_matrices}. In terms of precision of classification, all networks perform relatively well. However, the 3D convolutional network and the custom convolutional LSTM network were marginally more effective, achieving a higher accuracy and correctly classifying more challenging event streams. Where this is most apparent is when classifying numbers such as 3, since with the base convolutional LSTM network these were sometime misclassified as an 8 or 9.

The performance of each network on the frame integrated NMNIST was better overall than with the intensity reconstructed equivalent (as is evident by the superior f-score of the classification networks), which may be due to the fact that temporal information was not lost, as was the case in the reconstructed video stream. Due to the poor performance of the intensity reconstructions overall, it was outperformed in most, if not all cases.

\begin{figure}[htb]%
    \centering
    \subfloat[\centering]{{\includegraphics[width=0.4\textwidth]{testingandresults/images/c_matrix_nmnist_conv3d.png}}}%
    \qquad
    \subfloat[\centering]{{\includegraphics[width=0.4\textwidth]{testingandresults/images/c_matrix_nmnist_convlstm.png}}}%
    \qquad
    \subfloat[\centering]{{\includegraphics[width=0.4\textwidth]{testingandresults/images/c_matrix_nmnist_custom_convlstm.png}}}%
    \caption{Confusion matrices for frame-integrated NMNIST classification with various networks; \textbf{(a)} conv3D, \textbf{(b)} convLSTM, \textbf{(c)} custom convLSTM.}%
    \label{fig:nmnist_c_matrices}%
\end{figure}

\Cref{tab:conv3d_nmnist_evaluation_metrics}, \cref{tab:conv_lstm_nmnist_evaluation_metrics}, and \cref{tab:custom_conv_lstm_nmnist_evaluation_metrics} show the performance evaluation of each of the networks on the frame-integrated NMNIST dataset in more detail. Once again the network has trouble classifying the difficult classes. We can see that the performance suffers more when compared to the networks analysing the integrated frames. Especially in the NMNIST dataset the edges are very significant for distinguishing between each class, which integrated frames excel at highlighting. The additional details added to the frames by E2VID seem to only hinder performance.

\vspace{10pt}

The confusion matrices for the classification of the frame-integrated DVS128 Gesture dataset can be seen in \cref{fig:dvs128_c_matrices}. The variance in performance between the networks is more apparent in this dataset than for the NMNIST dataset. The reason for this is that not only is object detection (of the human body) being undertaken by the system, but also action recognition across frames. It is evident that actions 4 and 5 (right arm clockwise and right arm counter-clockwise), as well as actions 6 and 7 (left arm clockwise and left arm counter-clockwise), were often misclassified as one another. The reason for this was that with length 20 frame integrated videos the fast rotational movement results in frames that look very similar in both directions \color{red} TODO: Maybe try to get some pictures to prove this \color{black}. This could be remedied by having more than 20 frames for each integrated video sequence. Other mistakes were more often made with the 3D convolutional network. For example 10 (air guitar) was quite often predicted for cases of classes 8 and 9 (air roll and air drums). 

\begin{figure}[htb]%
    \centering
    \subfloat[\centering]{{\includegraphics[width=0.4\textwidth]{testingandresults/images/c_matrix_dvs128_conv3d.png}}}%
    \qquad
    \subfloat[\centering]{{\includegraphics[width=0.4\textwidth]{testingandresults/images/c_matrix_dvs128_convlstm.png}}}%
    \qquad
    \subfloat[\centering]{{\includegraphics[width=0.4\textwidth]{testingandresults/images/c_matrix_dvs128_custom_convlstm.png}}}%
    \caption{Confusion matrices for frame-integrated  DVS128 Gesure classification with various networks; \textbf{(a)} conv3D, \textbf{(b)} convLSTM, \textbf{(c)} custom convLSTM.}%
    \label{fig:dvs128_c_matrices}%
\end{figure}

\Cref{tab:conv3d_dvs128_evaluation_metrics}, \cref{tab:conv_lstm_dvs128_evaluation_metrics}, and \cref{tab:custom_conv_lstm_dvs128_evaluation_metrics} show the performance evaluation of each of the networks on the frame-integrated DVS128 Gesture dataset in more detail.

\subsection{Custom Frame Integration}

Some tests were undertaken using the custom frame integration technique on the DVS128 Gesture dataset. With this method, information can be stored to the same, or even smaller, temporal resolution in a lower channel image. \cref{fig:frame_integration_comparison} shows the left arm clockwise hand rotation gesture represented using the two techniques. It is clear that the encoding is more efficient, since only single-channelled greyscale images are required with the custom integration method, and more fine-grained temporal information is retained since the intensity of the pixels signifies the number of events occurring in any given pixel. For example, in \textbf{(b)} it is possible to visually know where the position of the arm is, as well as the movement it was going through without having to create too many individual integrated frames. We can see that when 20 integrated frames were created for the NMNIST dataset with the classical method, the motion is visible in one large blur, with no clear indication of the position of the arm.

\begin{figure}[htb]%
    \centering
    \subfloat[\centering]{{\includegraphics[width=0.18\textwidth]{testingandresults/images/classic_frame_integration.png}}}%
    \qquad
    \subfloat[\centering]{{\includegraphics[width=0.18\textwidth]{testingandresults/images/custom_frame_integration.png}}}%
    \caption{Frame integration of arm rotation gesture from DVS128 Gesture dataset with two different methods; \textbf{(a)} classic frame integration, and \textbf{(b)} custom frame integration.}%
    \label{fig:frame_integration_comparison}%
\end{figure}

\color{red} TODO: Write some experiments for comparison of the two techniques. \color{black} 

\subsubsection{Classification Results}

The confusion matrices for the classification of the frame-integrated  NMNIST dataset with the custom frame integration method can be seen in \cref{fig:nmnist_custom_frame_c_matrices}. \color{red} TODO: Write more about this and redo confusion matrices \color{black}.

\begin{figure}[htb]%
    \centering
    \subfloat[\centering]{{\includegraphics[width=0.4\textwidth]{testingandresults/images/c_matrix_nmnist_custom_frame_conv3d.png}}}%
    \qquad
    \subfloat[\centering]{{\includegraphics[width=0.4\textwidth]{testingandresults/images/c_matrix_nmnist_custom_frame_convlstm.png}}}%
    \qquad
    \subfloat[\centering]{{\includegraphics[width=0.4\textwidth]{testingandresults/images/c_matrix_nmnist_custom_frame_custom_convlstm.png}}}%
    \caption{Confusion matrices for custom frame-integrated NMNIST classification with various networks; \textbf{(a)} conv3D, \textbf{(b)} convLSTM, \textbf{(c)} custom convLSTM.}%
    \label{fig:nmnist_custom_frame_c_matrices}%
\end{figure}

\Cref{tab:conv3d_nmnist_custom_frame_evaluation_metrics}, \cref{tab:conv_lstm_nmnist_custom_frame_evaluation_metrics}, and \cref{tab:custom_conv_lstm_nmnist_custom_frame_evaluation_metrics} show the performance evaluation of each of the networks on the custom frame-integrated NMNIST dataset in more detail.

\color{red} TODO: Get result for custom frame-integrated DVS128 Gesture. \color{black}

\section{Comparison of Classification Networks}

\begin{table}[htb]
    \centering
    \begin{tabular}{|| c | l | c | c | c ||}
        \hline
        \multicolumn{2}{|| c |}{Dataset} & Conv3D & ConvLSTM & Custom ConvLSTM \\
        \hline \hline
        \multirow{3}{*}{NMNIST} & Intensity Reconstructed & 86.67\% & 79.60\% & 83.20\%\\
        \cline{2-5}
         & Frame Integrated & 98.74\% & 97.64\% & 97.70\% \\
        \cline{2-5}
        & Custom Frame Integrated & 98.22\% & 97.48\% & 97.88\% \\
        \hline
        \multirow{3}{*}{DVS128 Gesture} & Intensity Reconstructed & 63.26\% & - & 82.95\% \\
        \cline{2-5}
         & Frame Integrated & 69.44\% & 71.53\% & 76.39\% \\
        \cline{2-5}
         & Custom Frame Integrated & 69.32\% & 72.80\% & 80.14\% \\
        \hline
    \end{tabular}
    \caption{A table showing classification accuracies of various models.}
    \label{tab:network_performances}
\end{table}

The full table of classification accuracies can be seen in \cref{tab:network_performances}. Here, the different strengths of the various networks becomes apparent in this comparison. For the simple object classification task on the NMNIST dataset, the higher capacity (i.e., high number of trainable parameters) 3D convolutional network outperformed the others. The dataset's scale is small enough for the network to learn all possible features and get a good classification accuracy.

For the more complex cases of gesture recognition, however, the performance of this network lagged behind the recurrent LSTM networks (especially the custom LSTM network). The reason for this is that though it is excellent at detecting spacial patterns in each frame, it is less able to detect the temporal patterns prevalent in the gestures. For these cases the custom convLSTM network shone, outperforming the 3D convolution network by ~20\%. The 3D convolutional network could no longer memorise enough features to accurately classify all gestures, and even over-fit, leading to lower accuracies on the unseen testing data.

\begin{figure}[htb]
    \centering
    \includegraphics[width=0.8\textwidth]{testingandresults/images/training_time_and_trainable_parameters.png}
    \caption{A figure showing the training times and number of trainable parameters of each network.}
    \label{fig:training_times_and_trainable_parameters}
\end{figure}

The training times for each of the networks on the datasets can be seen in \cref{fig:training_times_and_trainable_parameters}. It is evident that the networks with a higher capacity had higher training times. The 3D convolution network and custom convLSTM networks in particular had a longer training times due to having many more parameters to optimise via back-propagation. It should be noted that for the networks in missing classification accuracies in \cref{tab:network_performances} the system often encountered an Out Of Memory (OOM) error when attempting create a tensor of larger batch size. For these cases a smaller batch size was used, resulting in slightly higher training times, as well as possible skewed results. If even this was not sufficient, results were not taken for these cases.

\section{Comparison of Event Analysis Pipelines}

Interestingly, there are some discrepancies in the performances of the two implemented pipelines when they are doing simple object detection as opposed to more complex tasks such as gesture recognition. Some comments about this discrepancy are given in this section.

When comparing the two-phase video reconstruction pipeline to the direct event classification pipeline, the results show that for the NMNIST dataset, the direct event analysis leads to superior classification accuracies. The reason for this could be the loss of some high frequency data when creating intensity reconstructions with event data. The reason for this is that the created video is of a certain frame-rate, which is much lower than the temporal resolution of events captured by event cameras (which are on the $ \mu s $ level scale). There is, however, scope to still use the reconstruction pipeline without losing too much temporal information. For this the number of events per input to the E2VID network can be reduced so that the resulting video is of a higher frame-rate. However, this would require for the network to be retrained to achieve good results on the data with fewer data-points per frame.

Moreover, in the case of NMNIST, the reconstructions were not too dissimilar to the integrated frames. In both pipelines the input was similar to an edge map, and so the performance difference was less noticeable. This may be because of the small size of the dataset frames, and so for better reconstruction performance the E2VID network would have to be re-trained. Overall the performance of the two-phase pipeline was lower than working directly on the integrated frames since even though both inputs were like edge maps, the integrated frames retained far more temporal data.

For the DVS128 Gesture dataset, the performance difference between the two pipelines was more noticeable. In this case the reconstructions were much more realistic and well-defined, allowing for the classification networks to more easily find patterns in the videos. Unlike in NMNIST, where the direct event classification pipeline outperformed the two-phase intensity reconstruction pipeline, for gesture recognition the opposite was true. The reconstructions were of particularly good quality for the DVS128 Gesture dataset, allowing for the classification networks to gather useful patterns. The reconstructions were also similar for all lighting conditions, meaning performance was consistently strong. 

It is important to note, however, that the size of the dataset was significantly larger when passed through the reconstruction network. The size difference was in the order of 10 times larger, which may be an indication of why the performance was higher. With a higher number of integrated frames per sample the discrepancy in performance may be smaller.

Overall, for more complex tasks such as gesture recognition, video reconstruction had better performance than with more simple classification problems. It is fair to assume this trend extrapolates to more complex classification tasks, since the intensity reconstructions add more detail to frames than are available in the events they process. As well as this, when tested on the Event Camera dataset, the reconstructions were much clearer visually than the respective video frames. More complex details from the environment were also reconstructed, meaning the there is more information held in the reconstructed frames than in the events they were generated from. This could be incredibly beneficial for more complex and fine-grained classification tasks.

\section{Evaluation Spiking Neural Network Conversion}

\subsection{Converted 3D Convolutional Network}

As mentioned in \cref{sssec:firing_rate_scaling}, there is a compromise to be made when attempting to scale neuron firing rates to get a better accuracy. When firing rates are scaled past a certain level, the network becomes equivalent to a non-spiking network. This would mean the loss of the benefits of sparse encoding the the SNN. The effect of over-scaling firing rate can be seen in \cref{fig:over_scaling_firing_rate}, where a very high firing rate resulted in a network very similar to a non-spiking neural network.

\begin{figure}[htb]%
    \centering
    \subfloat[\centering]{{\includegraphics[width=0.7\textwidth]{testingandresults/images/overscaled_firing_rate.png}}}%
    \qquad
    \subfloat[\centering]{{\includegraphics[width=0.7\textwidth]{testingandresults/images/non_spiking_comparison.png}}}%
    \caption{Comparison of spiking neural network with firing rate over-scaled \textbf{(a)} and a the equivalent non-spiking neural network \textbf{(b)}.}%
    \label{fig:over_scaling_firing_rate}%
\end{figure}

The relationship between firing rates and performance can be seen in \cref{fig:spiking_perfromace_with_varying_spike_rate}, where the accuracy got closer to the non-spiking network as the spiking rate was artificially inflated to infinity. It was found that for the conv3D network, comparable performance could be obtained while keeping the data-flow through each network quite sparse. Close to 100\% relative performance could be reached with an average neuron spiking rate of less than 1Hz for both the NMNIST and DVS128 datasets. Especially for the DVS128 dataset, with very few average neuron spikes, the network achieved good relative performance. This is a good indication that a fully neuromorphic system is possible, allowing for efficient and low-powered classification systems.

\begin{figure}[htb]
    \centering
    \includegraphics[width=0.8\textwidth]{testingandresults/images/spiking_perfromace_with_varying_spike_rate.png}
    \caption{A figure showing the performance of the spiking 3D conv network relative to its non-spiking equivalent with varying spike rates.}
    \label{fig:spiking_perfromace_with_varying_spike_rate}
\end{figure}

\subsection{Legendre Memory Unit Network}

\begin{table}[htb]
    \centering
    \begin{tabular}{|| c | c | c ||}
        \hline
        Network     & NMNIST & DVS128 Gesture \\
        \hline \hline
        Non-spiking LMU        &  90.2\%    &    60\%  \\
        \hline
        Spiking LMU        &  60\%    &      \\
        \hline
    \end{tabular}
    \caption{A table showing the performance of LMU networks on various datasets.}
    \label{tab:lmu_performance}
\end{table}

\cref{tab:lmu_performance} shows the promise of LMUs as a recurrent module for spiking neural networks. The performance before SNN conversion is similar to LSTM networks, and could feasibly improved with convolutional elements as was done with the convLSTM layer.

\color{red} TODO: Write about difficulties with spiking LSTM and experiments with LMU network. \color{black}
\chapter{Conclusion and Further Work} \label{chap:conclusion_and_further_work}

To conclude, the testing done with the two classification pipelines shows that there are significant benefits to working within the neuromorphic paradigm as opposed to classic frame-based cameras. With the same networks, performance appears to be superior when analysing event streams using classical frame-integration, as opposed to video frames for simple object recognition and classification tasks. Secondly, the two-phase intensity reconstruction pipelines show promise, being able to harness the wealth of existing networks that exist for video classification. 

The performance with the same classification networks directly on the events rather than the video reconstructions tended to have better performance for simple object recognition and classification tasks. This is perhaps due to the loss of high-frequency data and when creating the intensity maps. However, this was not the case for more complex tasks such as gesture recognition. In this case the reconstructions were found to be of superior quality (perhaps due to the size and nature of the dataset used), allowing for better analysis of the recorded gestures. It is fair to assume that this is due to the larger size of the DVS128 Gesture dataset, which is a more appropriate input for the reconstruction algorithm used.

As well as this, it was found that for analysing motion in event streams, the custom frame-integration method outlined in this project achieves the best results. In gesture recognition tasks it even outperformed the video reconstruction pipeline. This performance increase cannot be seen for object recognition, however, since temporal information is not as useful in this case. For example, on the NMNIST dataset the accuracy of this novel frame-integration method was found to be the same or worse than with classical frame-integration.

The reconstructions show that the benefits of event-based cameras can be retained even when working with intensity videos, as the reconstruction networks produce high fidelity outputs even when recording videos with lots of motion or high dynamic range of light intensities. This means that the reconstructed videos are often more appropriate for classification that if a video was taken with a frame-based camera. This is also true for integrated frames, which exploit the same benefits from event data.

Finally, testing the conversion of a 3D convolutional network showed that equivalent performance could be attained, while harnessing the benefits of spiking neural networks. Close to 100\% of the performance of the non-spiking equivalent was achieved while having a spiking rate of below 1Hz. This is beneficial since the information being passed between the layers of the neural network are now sparse, meaning less communication needs to take place overall to analyse the data. This is therefore a much less power intensive, and more computationally efficient encoding affording itself to the use on low-powered chips and for real-time processing. This is especially true of the fully neuromorphic system detailed in this report, since the input is from a event-based camera, which in itself is much more power-efficient than frame-based cameras. The Legendre Memory Unit also shows promise as a good way of implemented a recurrent unit in a spiking neural network, however more work needs to be done to tune the performance.

\section{Further Work}

Having reached the end of the project, it is important to note the limitations (such as computing power and available time) that need to be acknowledged. As such, there are some key areas in which there is scope for further research and development. These include;

\begin{enumerate}
    \item With a more capable setup and time to test the networks, there is the opportunity to create more intensive networks. As well as this larger batch sizes can be used for more stable training with higher RAM available.
    \item Compare the performance of the reconstructed pipeline like-for-like against a video stream from a frame-based camera. This has been omitted from this project due to time constraints and lack of available datasets.
    \item Change E2VID parameters so that resulting reconstructed video is of a higher framerate.
    \item Test the performance of the described networks and pipelines on more readily available datasets to make sure that the findings apply to other environments. Intensity reconstructions may be beneficial for more complex tasks.
    \item Look more at the use of natural language processing techniques such as Bag of Words and Support Vector Machines when working with event streams.
    \item Create more spiking neural networks, perhaps from scratch rather than converting regular ANNs. Further testing needs to be done in this area.
    \item Tune the LMU to work well on the given datasets.
    \item Test performance of spiking neural networks on intensity reconstructed video streams.
    \item Test the performance of the end-to-end event classification networks when frame-integration is done asynchronously as well as synchronously (as described in \cref{ssec:frame_integration}). 
    \item The effect of having smaller time-slices per integrated frame need to be tested, where the data has an equivalent size to the E2VID output.
    \item Create spiking neural networks based on more complex ANN networks. This involves creating approximations to other, more complex layers such as LSTM and GRU layers.
    \item Implement spiking neural networks on neuromorphic hardware. This also opens up the possibility of real-time inference.
    \item Create a spiking version of the intensity reconstruction algorithms to check if they perform well.
\end{enumerate}

\section{Similar External Work}

A method, similar to the custom frame integration proposed in \cref{sssec:custom_frame_integration_implementation}, that aimed to preserve temporal resolution was proposed by Lo\"ic Cordone \textit{et al.}, who proposed `voxel cubes'\cite{MiniVovelCubes}. For this method there were still binary events in every channel, but there were more than two channels so that the events in each time-slice could be subdivided into each channel. When compared to this method, the temporal information can be stored in the same way with the novel method proposed in this project, while keeping data size small since it is just a one-channel image. 

More work by Lo\"ic Cordone \textit{et al.}\cite{OtherSnnWork}, shows testing with sparse SNN architectures (using surrogate training as done in this report) achieving similar performance on the DVS dataset to the networks outlined in this report. In their paper a 2D spiking convolutional neural network is compared to a 3D convolutional network, and the training and inference times are outlines to show the efficiency of spiking encoding.

\appendix
\chapter{Classification Model Architectures}

\begin{lstlisting}[language=Bash,caption={Overview of layers in Convolutional LTSM network.},label={lst:conv_lstm_layers},numbers=none,float=htb]
_________________________________________________________________
Layer (type)                Output Shape              Param #   
=================================================================
conv_lstm2d_6 (ConvLSTM2D)  (None, 8, 34, 34, 64)     150016    
                                                                
max_pooling3d_4 (MaxPooling  (None, 8, 17, 17, 64)    0         
3D)                                                             
                                                                
batch_normalization_6 (Batc  (None, 8, 17, 17, 64)    256       
hNormalization)                                                 
                                                                
conv_lstm2d_7 (ConvLSTM2D)  (None, 8, 17, 17, 32)     110720    
                                                                
max_pooling3d_5 (MaxPooling  (None, 8, 9, 9, 32)      0         
3D)                                                             
                                                                
batch_normalization_7 (Batc  (None, 8, 9, 9, 32)      128       
hNormalization)                                                 
                                                                
conv_lstm2d_8 (ConvLSTM2D)  (None, 9, 9, 16)          27712     
                                                                
max_pooling2d_2 (MaxPooling  (None, 5, 5, 16)         0         
2D)                                                             
                                                                
batch_normalization_8 (Batc  (None, 5, 5, 16)         64        
hNormalization)                                                 
                                                                
flatten_2 (Flatten)         (None, 400)               0         
                                                                
dense_4 (Dense)             (None, 256)               102656    
                                                                
dense_5 (Dense)             (None, 10)                2570      
                                                                
=================================================================
Total params: 394,122
Trainable params: 393,898
Non-trainable params: 224
_________________________________________________________________
\end{lstlisting}

\begin{lstlisting}[language=Bash,caption={Overview of layers in 3D convolutional network},label={lst:3d_conv_layers},numbers=none,float=htb]
_________________________________________________________________
Layer (type)                Output Shape              Param #   
=================================================================
conv3d_10 (Conv3D)          (None, 8, 34, 34, 32)     4032      
                                                                
batch_normalization_12 (Bat  (None, 8, 34, 34, 32)    128       
chNormalization)                                                
                                                                
conv3d_11 (Conv3D)          (None, 8, 34, 34, 32)     128032    
                                                                
batch_normalization_13 (Bat  (None, 8, 34, 34, 32)    128       
chNormalization)                                                
                                                                
max_pooling3d_6 (MaxPooling  (None, 4, 17, 17, 32)    0         
3D)                                                             
                                                                
dropout_8 (Dropout)         (None, 4, 17, 17, 32)     0         
                                                                
conv3d_12 (Conv3D)          (None, 4, 17, 17, 64)     256064    
                                                                
batch_normalization_14 (Bat  (None, 4, 17, 17, 64)    256       
chNormalization)                                                
                                                                
max_pooling3d_7 (MaxPooling  (None, 2, 9, 9, 64)      0         
3D)                                                             
                                                                
dropout_9 (Dropout)         (None, 2, 9, 9, 64)       0         
                                                                
conv3d_13 (Conv3D)          (None, 2, 9, 9, 128)      1024128   
                                                                
batch_normalization_15 (Bat  (None, 2, 9, 9, 128)     512       
chNormalization)                                                
                                                                
conv3d_14 (Conv3D)          (None, 2, 9, 9, 128)      2048128   
                                                                
batch_normalization_16 (Bat  (None, 2, 9, 9, 128)     512       
chNormalization)                                                
                                                                
max_pooling3d_8 (MaxPooling  (None, 1, 5, 5, 128)     0         
3D)                                                             
                                                                
dropout_10 (Dropout)        (None, 1, 5, 5, 128)      0         
                                                                
flatten_2 (Flatten)         (None, 3200)              0         
                                                                
dense_4 (Dense)             (None, 128)               409728    
                                                                
batch_normalization_17 (Bat  (None, 128)              512       
chNormalization)                                                
                                                                
dropout_11 (Dropout)        (None, 128)               0         
                                                                
dense_5 (Dense)             (None, 10)                1290      
                                                                
activation_2 (Activation)   (None, 10)                0         
                                                                
=================================================================
Total params: 3,873,450
Trainable params: 3,872,426
Non-trainable params: 1,024
_________________________________________________________________
\end{lstlisting}

\begin{lstlisting}[language=Bash,caption={Overview of layers in Custom Convolutional LTSM network.},label={lst:custom_conv_lstm_layers},numbers=none,float=htb]
_________________________________________________________________
Layer (type)                Output Shape              Param #   
=================================================================
time_distributed_3 (TimeDis  (None, 8, 2048)          4887936   
tributed)                                                       
                                                                
gru_3 (GRU)                 (None, 64)                405888    
                                                                
dense_25 (Dense)            (None, 1024)              66560     
                                                                
dropout_9 (Dropout)         (None, 1024)              0         
                                                                
dense_26 (Dense)            (None, 512)               524800    
                                                                
dropout_10 (Dropout)        (None, 512)               0         
                                                                
dense_27 (Dense)            (None, 128)               65664     
                                                                
dropout_11 (Dropout)        (None, 128)               0         
                                                                
dense_28 (Dense)            (None, 64)                8256      
                                                                
dense_29 (Dense)            (None, 10)                650       
                                                                
=================================================================
Total params: 5,959,754
Trainable params: 5,959,754
Non-trainable params: 0
_________________________________________________________________
\end{lstlisting}

\begin{lstlisting}[language=Bash,caption={Overview of layers in Custom 2D Convolutional network built into the Custom LSTM Network in \cref{lst:custom_conv_lstm_layers}.},label={lst:custom_conv_2d_layers},numbers=none,float=htb]
_________________________________________________________________
Layer (type)                Output Shape              Param #   
=================================================================
conv2d_3 (Conv2D)           (None, 34, 34, 128)       640       
                                                                
activation_3 (Activation)   (None, 34, 34, 128)       0         
                                                                
max_pooling2d_4 (MaxPooling  (None, 17, 17, 128)      0         
2D)                                                             
                                                                
conv2d_4 (Conv2D)           (None, 17, 17, 256)       131328    
                                                                
activation_4 (Activation)   (None, 17, 17, 256)       0         
                                                                
max_pooling2d_5 (MaxPooling  (None, 8, 8, 256)        0         
2D)                                                             
                                                                
conv2d_5 (Conv2D)           (None, 8, 8, 512)         524800    
                                                                
activation_5 (Activation)   (None, 8, 8, 512)         0         
                                                                
max_pooling2d_6 (MaxPooling  (None, 4, 4, 512)        0         
2D)                                                             
                                                                
flatten_2 (Flatten)         (None, 8192)              0         
                                                                
dense_7 (Dense)             (None, 256)               2097408   
                                                                
activation_6 (Activation)   (None, 256)               0         
                                                                
dense_8 (Dense)             (None, 10)                2570      
                                                                
activation_7 (Activation)   (None, 10)                0         
                                                                
=================================================================
Total params: 2,756,746
Trainable params: 2,756,746
Non-trainable params: 0
_________________________________________________________________
\end{lstlisting}
\appendix
\chapter{Evaluation Metrics of Each Trained Model}

\begin{table}[htb]
    \centering
    \begin{tabular}{|| c | c | c | c | c ||}
        \hline
             & Precision & Recall & F1-score & Samples \\
        \hline \hline
        0            & 1.00  & 0.97  & 0.98 & 500 \\
        \hline
        1            & 0.99  & 1.00  & 0.99 & 500 \\
        \hline
        2            & 0.99  & 0.99  & 0.99 & 500 \\
        \hline
        3            & 0.99  & 0.99  & 0.99 & 500 \\
        \hline
        4            & 0.99  & 0.97  & 0.98 & 500 \\
        \hline
        5            & 1.00  & 0.98  & 0.99 & 500 \\
        \hline
        6            & 0.97  & 1.00  & 0.98 & 500 \\
        \hline
        7            & 0.98  & 0.97  & 0.98 & 500 \\
        \hline
        8            & 0.97  & 0.97  & 0.97 & 500 \\
        \hline
        9            & 0.95  & 0.98  & 0.96 & 500 \\
        \hline
        macro avg    & 0.98  & 0.98  & 0.98 & 5000 \\
        \hline
        weighted avg & 0.98  & 0.98  & 0.98 & 5000 \\
        \hline
    \end{tabular}
    \caption{A table showing classification evaluation metrics of 3D convolutional network on the frame-integrated NMNIST dataset.}
    \label{tab:conv3d_nmnist_evaluation_metrics}
\end{table}

\begin{table}[htb]
    \centering
    \begin{tabular}{|| c | c | c | c | c ||}
        \hline
             & Precision & Recall & F1-score & Samples \\
        \hline \hline
        0            & 0.99 & 0.99 & 0.99 & 500  \\
        \hline
        1            & 0.98 & 1.00 & 0.99 & 500  \\
        \hline
        2            & 0.97 & 0.99 & 0.98 & 500  \\
        \hline
        3            & 0.94 & 0.99 & 0.96 & 500  \\
        \hline
        4            & 1.00 & 0.98 & 0.99 & 500  \\
        \hline
        5            & 0.96 & 0.99 & 0.98 & 500  \\
        \hline
        6            & 0.99 & 0.99 & 0.99 & 500  \\
        \hline
        7            & 0.96 & 0.98 & 0.97 & 500  \\
        \hline
        8            & 0.98 & 0.91 & 0.95 & 500  \\
        \hline
        9            & 0.97 & 0.94 & 0.96 & 500  \\
        \hline
        macro avg    & 0.98 & 0.97 & 0.97 & 5000 \\
        \hline
        weighted avg & 0.98 & 0.97 & 0.97 & 5000 \\
        \hline
    \end{tabular}
    \caption{A table showing classification evaluation metrics of convolutional LSTM network on the frame-integrated NMNIST dataset.}
    \label{tab:conv_lstm_nmnist_evaluation_metrics}
\end{table}

\begin{table}[htb]
    \centering
    \begin{tabular}{|| c | c | c | c | c ||}
        \hline
             & Precision & Recall & F1-score & Samples \\
        \hline \hline
        0            & 0.98 & 0.98 & 0.98 & 500  \\
        \hline
        1            & 0.98 & 1.00 & 0.99 & 500  \\
        \hline
        2            & 0.98 & 0.99 & 0.98 & 500  \\
        \hline
        3            & 0.98 & 0.97 & 0.97 & 500  \\
        \hline
        4            & 0.99 & 0.98 & 0.98 & 500  \\
        \hline
        5            & 0.98 & 0.98 & 0.98 & 500  \\
        \hline
        6            & 0.97 & 0.99 & 0.98 & 500  \\
        \hline
        7            & 0.96 & 0.98 & 0.97 & 500  \\
        \hline
        8            & 0.97 & 0.94 & 0.96 & 500  \\
        \hline
        9            & 0.98 & 0.94 & 0.96 & 500  \\
        \hline
        macro avg    & 0.98 & 0.98 & 0.98 & 5000 \\
        \hline
        weighted avg & 0.98 & 0.98 & 0.98 & 5000 \\
        \hline
    \end{tabular}
    \caption{A table showing classification evaluation metrics of custom convolutional LSTM network on the frame-integrated NMNIST dataset.}
    \label{tab:custom_conv_lstm_nmnist_evaluation_metrics}
\end{table}

\begin{table}[htb]
    \centering
    \begin{tabular}{|| c | c | c | c | c ||}
        \hline
             & Precision & Recall & F1-score & Samples \\
        \hline \hline
        0            & 0.94 & 0.96 & 0.95 & 75  \\
        \hline
        1            & 0.95 & 0.96 & 0.95 & 75  \\
        \hline
        2            & 0.85 & 0.93 & 0.89 & 75  \\
        \hline
        3            & 0.84 & 0.71 & 0.77 & 75  \\
        \hline
        4            & 0.93 & 0.84 & 0.88 & 75  \\
        \hline
        5            & 0.71 & 0.87 & 0.78 & 75  \\
        \hline
        6            & 0.87 & 0.91 & 0.89 & 75  \\
        \hline
        7            & 0.91 & 0.85 & 0.88 & 75  \\
        \hline
        8            & 0.86 & 0.73 & 0.79 & 75  \\
        \hline
        9            & 0.85 & 0.91 & 0.88 & 75  \\
        \hline
        macro avg    & 0.87 & 0.87 & 0.87 & 750 \\
        \hline
        weighted avg & 0.87 & 0.87 & 0.87 & 750 \\
        \hline
    \end{tabular}
    \caption{A table showing classification evaluation metrics of 3D convolutional network on the reconstructed NMNIST dataset.}
    \label{tab:conv3d_nmnist_recon_evaluation_metrics}
\end{table}

\begin{table}[htb]
    \centering
    \begin{tabular}{|| c | c | c | c | c ||}
        \hline
             & Precision & Recall & F1-score & Samples \\
        \hline \hline
        0            & 0.83 & 0.92 & 0.87 & 75  \\
        \hline
        1            & 0.96 & 0.96 & 0.96 & 75  \\
        \hline
        2            & 0.82 & 0.91 & 0.86 & 75  \\
        \hline
        3            & 0.77 & 0.57 & 0.66 & 75  \\
        \hline
        4            & 0.74 & 0.72 & 0.73 & 75  \\
        \hline
        5            & 0.67 & 0.71 & 0.69 & 75  \\
        \hline
        6            & 0.88 & 0.85 & 0.86 & 75  \\
        \hline
        7            & 0.85 & 0.84 & 0.85 & 75  \\
        \hline
        8            & 0.74 & 0.64 & 0.69 & 75  \\
        \hline
        9            & 0.71 & 0.84 & 0.77 & 75  \\
        \hline
        macro avg    & 0.80 & 0.80 & 0.79 & 750 \\
        \hline
        weighted avg & 0.80 & 0.80 & 0.79 & 750 \\
        \hline
    \end{tabular}
    \caption{A table showing classification evaluation metrics of convolutional LSTM network on the reconstructed NMNIST dataset.}
    \label{tab:conv_lstm_nmnist_recon_evaluation_metrics}
\end{table}

\begin{table}[htb]
    \centering
    \begin{tabular}{|| c | c | c | c | c ||}
        \hline
             & Precision & Recall & F1-score & Samples \\
        \hline \hline
        0            & 0.88 & 0.95 & 0.91 & 75  \\
        \hline
        1            & 0.99 & 0.92 & 0.95 & 75  \\
        \hline
        2            & 0.93 & 0.85 & 0.89 & 75  \\
        \hline
        3            & 0.79 & 0.76 & 0.78 & 75  \\
        \hline
        4            & 0.96 & 0.73 & 0.83 & 75  \\
        \hline
        5            & 0.71 & 0.81 & 0.76 & 75  \\
        \hline
        6            & 0.81 & 0.92 & 0.86 & 75  \\
        \hline
        7            & 0.82 & 0.84 & 0.83 & 75  \\
        \hline
        8            & 0.68 & 0.68 & 0.68 & 75  \\
        \hline
        9            & 0.82 & 0.85 & 0.84 & 75  \\
        \hline
        macro avg    & 0.84 & 0.83 & 0.83 & 750 \\
        \hline
        weighted avg & 0.84 & 0.83 & 0.83 & 750 \\
        \hline
    \end{tabular}
    \caption{A table showing classification evaluation metrics of custom convolutional LSTM network on the reconstructed NMNIST dataset.}
    \label{tab:custom_conv_lstm_nmnist_recon_evaluation_metrics}
\end{table}

\begin{table}[htb]
    \centering
    \begin{tabular}{|| c | c | c | c | c ||}
        \hline
            & Precision & Recall & F1-score & Samples \\
        \hline
        \hline
        0            & 0.55 & 0.88 & 0.68 & 24  \\
        \hline
        1            & 0.83 & 1.00 & 0.91 & 24  \\
        \hline
        2            & 0.77 & 0.83 & 0.80 & 24  \\
        \hline
        3            & 0.80 & 1.00 & 0.89 & 24  \\
        \hline
        4            & 0.60 & 0.75 & 0.67 & 24  \\
        \hline
        5            & 0.75 & 0.38 & 0.50 & 24  \\
        \hline
        6            & 0.80 & 0.33 & 0.47 & 24  \\
        \hline
        7            & 0.57 & 0.88 & 0.69 & 24  \\
        \hline
        8            & 0.75 & 0.79 & 0.77 & 48  \\
        \hline
        9            & 0.89 & 0.33 & 0.48 & 24  \\
        \hline
        10           & 0.94 & 0.62 & 0.75 & 24  \\
        \hline
        macro avg    & 0.75 & 0.71 & 0.69 & 288 \\
        \hline
        weighted avg & 0.75 & 0.72 & 0.70 & 288 \\
        \hline
    \end{tabular}
    \caption{A table showing classification evaluation metrics of 3D convolutional network on the frame-integrated DVS128 Gesture dataset.}
    \label{tab:conv3d_dvs128_evaluation_metrics}
\end{table}

\begin{table}[htb]
    \centering
    \begin{tabular}{|| c | c | c | c | c ||}
        \hline
             & Precision & Recall & F1-score & Samples \\
        \hline
        \hline
        0            & 0.55 & 0.88 & 0.68 & 24  \\
        \hline
        1            & 0.83 & 1.00 & 0.91 & 24  \\
        \hline
        2            & 0.77 & 0.83 & 0.80 & 24  \\
        \hline
        3            & 0.80 & 1.00 & 0.89 & 24  \\
        \hline
        4            & 0.60 & 0.75 & 0.67 & 24  \\
        \hline
        5            & 0.75 & 0.38 & 0.50 & 24  \\
        \hline
        6            & 0.80 & 0.33 & 0.47 & 24  \\
        \hline
        7            & 0.57 & 0.88 & 0.69 & 24  \\
        \hline
        8            & 0.75 & 0.79 & 0.77 & 48  \\
        \hline
        9            & 0.89 & 0.33 & 0.48 & 24  \\
        \hline
        10           & 0.94 & 0.62 & 0.75 & 24  \\
        \hline
        macro avg    & 0.75 & 0.71 & 0.69 & 288 \\
        \hline
        weighted avg & 0.75 & 0.72 & 0.70 & 288 \\
        \hline
    \end{tabular}
    \caption{A table showing classification evaluation metrics of convolutional LSTM network on the frame-integrated DVS128 Gesture dataset.}
    \label{tab:conv_lstm_dvs128_evaluation_metrics}
\end{table}

\begin{table}[htb]
    \centering
    \begin{tabular}{|| c | c | c | c | c ||}
        \hline
             & Precision & Recall & F1-score & Samples \\
        \hline
        \hline
        0            & 0.77 & 0.83 & 0.80 & 24  \\
        \hline
        1            & 0.92 & 0.96 & 0.94 & 24  \\
        \hline
        2            & 0.82 & 0.75 & 0.78 & 24  \\
        \hline
        3            & 0.91 & 0.88 & 0.89 & 24  \\
        \hline
        4            & 0.58 & 0.75 & 0.65 & 24  \\
        \hline
        5            & 0.70 & 0.58 & 0.64 & 24  \\
        \hline
        6            & 0.52 & 0.58 & 0.55 & 24  \\
        \hline
        7            & 0.54 & 0.54 & 0.54 & 24  \\
        \hline
        8            & 0.94 & 0.98 & 0.96 & 48  \\
        \hline
        9            & 0.70 & 0.67 & 0.68 & 24  \\
        \hline
        10           & 0.94 & 0.67 & 0.78 & 24  \\
        \hline
        macro avg    & 0.76 & 0.74 & 0.75 & 288 \\
        \hline
        weighted avg & 0.77 & 0.76 & 0.76 & 288 \\
        \hline
    \end{tabular}
    \caption{A table showing classification evaluation metrics of a convolutional LSTM on the frame-integratedDVS128 Gesture dataset.}
    \label{tab:custom_conv_lstm_dvs128_evaluation_metrics}
\end{table}

\begin{table}[htb]
    \centering
    \begin{tabular}{|| c | c | c | c | c ||}
        \hline
            & Precision & Recall & F1-score & Samples \\
        \hline \hline
        0            & 0.53 & 0.88 & 0.66 & 24  \\
        \hline
        1            & 0.85 & 0.96 & 0.90 & 24  \\
        \hline
        2            & 0.88 & 0.88 & 0.88 & 24  \\
        \hline
        3            & 0.43 & 0.92 & 0.59 & 24  \\
        \hline
        4            & 0.42 & 0.21 & 0.28 & 24  \\
        \hline
        5            & 0.61 & 0.58 & 0.60 & 24  \\
        \hline
        6            & 0.52 & 0.71 & 0.60 & 24  \\
        \hline
        7            & 0.76 & 0.60 & 0.67 & 48  \\
        \hline
        8            & 1.00 & 0.21 & 0.34 & 24  \\
        \hline
        9            & 0.91 & 0.42 & 0.57 & 24  \\
        \hline
        macro avg    & 0.69 & 0.64 & 0.61 & 264 \\
        \hline
        weighted avg & 0.70 & 0.63 & 0.61 & 264 \\
        \hline
    \end{tabular}
    \caption{A table showing classification evaluation metrics of 3D convolutional network on reconstructed DVS128 Gesture dataset.}
    \label{tab:conv3d_dvs128_recon_evaluation_metrics}
\end{table}

\begin{table}[htb]
    \centering
    \begin{tabular}{|| c | c | c | c | c ||}
        \hline
            & Precision & Recall & F1-score & Samples \\
        \hline \hline
        0            & 0.67 & 0.83 & 0.74 & 24  \\
        \hline
        1            & 0.86 & 1.00 & 0.92 & 24  \\
        \hline
        2            & 0.89 & 1.00 & 0.94 & 24  \\
        \hline
        3            & 0.85 & 0.46 & 0.59 & 24  \\
        \hline
        4            & 0.66 & 0.88 & 0.75 & 24  \\
        \hline
        5            & 0.95 & 0.79 & 0.86 & 24  \\
        \hline
        6            & 0.81 & 0.88 & 0.84 & 24  \\
        \hline
        7            & 0.88 & 0.92 & 0.90 & 48  \\
        \hline
        8            & 0.81 & 0.54 & 0.65 & 24  \\
        \hline
        9            & 1.00 & 0.92 & 0.96 & 24  \\
        \hline
        macro avg    & 0.84 & 0.82 & 0.82 & 264 \\
        \hline
        weighted avg & 0.84 & 0.83 & 0.82 & 264 \\
        \hline
    \end{tabular}
    \caption{A table showing classification evaluation metrics of custom convolutional LSTM network on reconstructed DVS128 Gesture dataset.}
    \label{tab:custom_conv_lstm_dvs128_recon_evaluation_metrics}
\end{table}
\chapter{Project Code Listing}

The full code listing can be found at the following link:

\vspace{10pt}

\url{https://github.com/TeDand/neuromorphic-classification}

\bibliographystyle{IEEEtranN}
\bibliography{bibs/bibliography}

\end{document}