\chapter{Evaluation} \label{chap:evaluation}

This chapter aims to follow on from the previous one which showed the results of each individual network. It serves as a high level overview of the performances of the end-to-end event classification pipeline and two-phase video reconstruction pipeline. There are also comparisons of each of the pipelines, allowing for the discerning their benefits and shortcomings.

\color{red} TODO: Write out the following;

\begin{itemize}
    % \item Loss of high frequency data
    % \item conventional networks built around frame based vision
    \item for NMNIST the small size of images means that reconstructions do no vary much from the rudimentary intensity maps (could be used as an illustration of how reconstruction works).
    \item Similar to denoising network CNN more easily able to find patterns in data
    \item gestures are more complicated to characterise, since the networks for NMIST can easily learn all possible features. As well as this the image size for NMNIST is very small
    % \item events tend to give edge map (does filling in the surroundings help?)
    \item intensity maps can be of much higher framerate since events are taken asynchronously
    \item gesture intensity maps equivalent to 100fps but retains high frequency information without too much noise
    \item most gesture recognition datasets involve removing backgrounds and focussing on the hand, our network automatically only sees moving objects
    \item gesture recognition robust for even harsh lighting conditions
\end{itemize}

\color{black}