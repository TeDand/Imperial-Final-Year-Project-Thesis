\chapter{Introduction}
% The introduction should set the scene and give a high-level problem statement/specification, so that after 
% reading the introduction the reader understands
% roughly what the problem is and what you intend to 
% do about it. Is the idea to write software, or develop 
% an algorithm, or produce hardware, or something 
% else?
% You should then highlight and summarise the most 
% interesting or important questions or problems that 
% your project addresses, and the broader context in 
% which those questions or problems are situated.
% Finally, you must briefly introduce the structure of 
% report (what you will cover in which chapters and how 
% these relate to each other). You don’t need to go into 
% any detail, the aim is to make sure the reader has an 
% idea about what will be discussed and in what order.

\section{Motivations}

Neuromorphic data is a novel representation of data that has overarching benefits that are yet to be fully explored in a plethora of applications. The many purported benefits of event-based cameras have the potential to revolutionise efficient, low-power computer vision due to their efficient encoding of information. Since the cameras are asynchronous in nature they allow for low-latency feeds with very little motion blur and other similar visual artefacts. As well as this their pixel structure allows for a very high dynamic range making the uses of these cameras even more apparent.

With the emergence of the IoT, sensors have become, or at the very least will become, ubiquitous in everyday life. These devices operate using power at a premium since they need to make full use of their limited batteries. This scarcity is further emphasised when attempting to do more power-intensive tasks. One such task is computer vision, which is becoming evermore prevalent as a means of human-computer interaction. Classical frame-based cameras are very power intensive, and do not allow for a low active duty cycle (i.e., allowing for device sleep/idle time). The usual way this issue is alleviated it by the system reacting to an event trigger. Such events may include things such as; motion, timing, acceleration or temperature. When compared to the function of event-based cameras, where event detection in built into the system, this can be seen as a stop-gap measure.

The high temporal resolution and dynamic range of event-based cameras also presents further benefits when compared to frame-based cameras. These features allow for videos to be represented in a very high fidelity format that preserves more data in fast-paced and brightly lit environments. This may lead to more reliable use of the data for computer vision and even other purposes.

\section{Objectives}

The objectives of this project were roughly divided into three two main sections; main objectives and extensions.

\subsection{Main Objectives}

\begin{itemize}
    \item Evaluate different Artificial Neural Network (ANN) models on neuromorphic data from external datasets.

          This allows for determining the performance of both traditional Neural Networks (NNs) as well the performance Spiking Neural Networks (SNNs) for comparison. These networks can take two main forms:
          \begin{itemize}
              \item Two networks sequentially processing data. The first would be converting reconstructing spiking data into an intensity video, and the second would be applying pre-existing computer vision networks to analyse the frame-based output.
              \item Having one network that takes the spiking data as input to directly carry out the intended function.
          \end{itemize}
          The system will initially be used to solve a classification task.
    \item Create practical set-up to experimentally record data.

          Using a neuromorphic camera, data can be experimentally captured to assess the performance of any built networks on more realistic unseen input data.
    \item Carry out Simultaneous Localisation and Mapping (SLAM) for a robot moving in an unknown trajectory.

          Use most efficient NN model to solve the more complex task of SLAM. The neuromorphic camera can be used to capture data from a room in the absence of large sets of labelled datasets.
\end{itemize}

\subsection{Challenges \& Fall-backs}

\begin{itemize}
    \item Only using existing datasets rather than practical set-up.

          Since there is sufficient amounts of existing data-sets, they can be split into training, validation and testing sets themselves. This eliminates the need to generate more data for the testing process.
    \item Rather than focusing on SLAM the emphasis may be on object detection/recognition

          The project has be segmented into individual milestones, and so even if the final one of a SLAM algorithm isn't achieved, it is easy to change the scope of the project to be simply a classification or gesture recognition network.
\end{itemize}

\section{Report Structure}

A brief outline of the report is as outlined:

\begin{enumerate}
    \item Background

          This section gives a background to the project subject. There are outlines of previous works in the field in order to highlight gaps in knowledge where more work can be done. It provides a good illustration of what issues there are and what value there would be in solving them.
    \item Implementation Plan

          This section has a list of planned objectives and their estimated completion date.
    \item Evaluation Plan

          This section presents a variety of ways in which the final project could be evaluated at each step. This allows for a comparison between existing solutions as well every iteration of solution given during the project.
    \item Ethical, Legal and Safety Plan

          Any ethical, legal and safety considerations taken into account during the project are given in this section.
\end{enumerate}